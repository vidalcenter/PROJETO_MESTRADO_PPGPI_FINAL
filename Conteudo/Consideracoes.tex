\chapter{CONSIDERAÇÕES FINAIS}


Com o desenvolvimento deste estudo, pode-se perceber que os objetivos propostos pelo Programa Empreenda Agro Sustentável foram contemplados na medida em que foram mobilizados vários estudantes de graduação das ciências agrárias e de outras áreas do conhecimento da UFS, em um desafio de criarem propostas de startups utilizando metodologias ativas numa fase de pré-aceleração. Nesse sentido, 15 equipes apresentaram seus Modelos de Negócio com bastante consistência, atraindo a atenção de investidores, ou se colocando para discussão com aceleradoras, que se dispuserem na busca por investidores. Desta maneira, as universidades e faculdades podem promover assistência de qualidade no fomento da autoeficácia e intenção empreendedora dos alunos.


Os resultados iniciais mostram que iniciativas que promovem a educação empreendedora e a percepção favorável de um ambiente universitário empresarial influenciam positivamente duas das três principais dimensões estudadas, que influenciam o surgimento da educação empreendedora nos alunos dos cursos das ciências agrárias, que são: Autoeficácia e Intenção Empreendedora. Outro aspecto de grande relevância foi a atratividade do Programa de muitos parceiros que se dispuseram a apoiar a realização de todas as etapas da jornada de diversas formas.

A melhoria da Educação Empreendedora no ensino superior, especialmente nos cursos das ciências agrárias, com ênfase, na prática, e no contato com os novos empreendimentos, pode contribuir diretamente para a formação de profissionais mais capazes de gerar novos negócios escaláveis, já que a intenção empreendedora em conjunto com a autoeficácia pode ser influenciada positivamente por programas educacionais, como foi apresentado nos resultados deste trabalho.

Ficou evidente nesta pesquisa que, o Programa Empreenda Agro Sustentável foi pioneiro nas conduções de ações com essa formação de conteúdos para a educação universitária das ciências agrarias no estado, é capaz de aumentar o desenvolvimento do comportamento empreendedor, mesmo em situações em que a Intenção empreendedora venha atuar como redutora do comportamento empreendedor tais como a participação familiar do aluno. A melhoria da intenção em empreender guiada pelas ferramentas educacionais durante a graduação, se mostra flexível positivamente por meio das contribuições diretas que este tipo de programa educacional promove como já sustentou a introdução deste artigo.

Os resultados desta pesquisa demostram que os participantes evoluem positivamente suas ambições empreendedoras e de desenvolvimento de novos negócios planejados na etapa de pré-aceleração, e encontram maior segurança para o próximo passo que é o de aceleração de seus planos de negócios, conquistando autonomia, buscando novas oportunidades, como profissionais proativos com destaque no mercado de trabalho.

Quanto observado o desenvolvimento de ferramentas auxiliares tais como o aplicativo Empreenda Agro Sustentável é possível observar que o projeto alcançou o propósito, que foi de servir como uma ferramenta para aprimorar os conteúdos ministrados sobre empreendedorismo sustentável no meio rural, fornecendo acesso a informações e ajudando a motivar o envolvimento com o auto-gerenciamento dos alunos participantes do programa. Esses objetivos foram atendidos adequadamente, oferecendo aos usuários participantes a oportunidade de revisar, reforçar e dispor de conteúdos direcionados.

O aplicativo também pode ser indicado para profissionais que tenham interesse em aprender mais sobre o desenvolvimento de negócios sustentáveis e escaláveis. O aplicativo já possui registro no Instituto Nacional de Propriedade Industrial INPI sob número de registro \textbf{BR 51 2019 002657 8} e disponível gratuitamente para testes na loja online \textit{Google Play Store}.


Como dificuldade encontrada para continuidade efetiva do projeto foi a pandemia causada pela Sars-CoV-2 mudando drasticamente o desenvolvimento e a forma da condução do programa, o comportamento recluso em todo o mundo  e suas organizações levaram aos alunos a terem que se adaptar às novas regas de segurança sanitárias. Este comportamento afetou drasticamente  processo de ensino e aprendizagem. Com a pandemia, o governo Brasil decidiu pela suspensão total de qualquer tipo de meio pedagógico que gerasse aglomeração, fato que resultou na interrupção das aulas de todos os alunos, que ficaram impossibilitados de retornar a universidades. Esta situação forçou o programa a adiar a continuidade presencial ano letivo de 2020, visando solucionar esta lacuna foi desenvolvido novos projetos de ensino de forma remota tais como o I Simpósio Agro Sustentável, O I Fórum Agro Sustentável e o Canal Empreenda AGROCAST, todos focados na manutenção da qualidade do ensino oferecido pelo projeto.

Para estudos futuros, é recomendável que seja realizada uma segregação dos alunos por cursos e faça-se uma pesquisa mais profunda relacionada ao método de estudo adotado pelos planos curriculares de cada curso, aliado ao entendimento das variáveis que compõe o constructo autoeficácia, intenção e participação familiar.

É oportuno observar que, para os próximos trabalhos é recomendável que a dimensão da motivação empreendedora seja estudada assim como, quais as iniciativas didáticas levam os alunos a buscar cursos e programas de extensão que tratam desta temática, compreender qual impulso do indivíduo para desenvolver um novo negócio, até que ele se torne o objeto da sua realização pessoal.

E por fim, é recomendável o desenvolvimento de uma pesquisa focada nos resultados de um projeto de aceleração, que permita avaliar os alunos participantes na fase de aceleração de suas ideias desenvolvidas durante o programa de natureza pré-aceleratório, possibilitando aferir os níveis de influência do programa após a fase proposta neste estudo.
