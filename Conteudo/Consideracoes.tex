\chapter{CONSIDERAÇÕES FINAIS}


Com o desenvolvimento deste estudo, pode-se perceber que, os objetivos propostos pelo Programa Empreenda Agro Sustentável foram contemplados na medida em que foram mobilizados vários estudantes de graduação as ciências agrarias e de outras áreas do conhecimento da UFS, em um desafio de criarem propostas de startups se utilizando de metodologias ativas, numa fase de pré-aceleração. Nesse sentido, 15 equipes apresentaram seus Modelos de Negócio com bastante consistência, atraindo a atenção de investidores, ou se colocando para discussão com aceleradoras, que se dispuserem na busca de investidores. Desta feita, as universidades e faculdades sempre podem fazer mais para fornecer assistência de qualidade no fomento da autoeficácia e intenção empreendedora dos alunos e de uma boa escolha de carreira, em geral.

Os resultados iniciais mostram que iniciativas que promovem a educação empreendedora e a percepção favorável de um ambiente universitário empresarial influenciam positivamente duas das três principais competências estudadas, que influenciam o surgimento da educação empreendedora nos alunos dos cursos das ciências agrárias, que são: Autoeficácia, Intenção Empreendedora. Outro aspecto de grande relevância foi a atratividade do Programa de muitos parceiros que se dispuseram a apoiar a realização de todas as etapas da jornada de diversas formas.

A melhoria da Educação Empreendedora no ensino superior, especialmente aos cursos das ciências agrárias, com ênfase, na prática, e no contato com os novos empreendimentos, pode contribuir diretamente para a formação de profissionais, mais capazes de gerar novos negócios escaláveis, já que a intenção empreendedora em conjunto com a autoeficácia pode ser influenciada para melhora por programas educacionais, como já sustentou os resultados deste trabalho.

Embora o apoio educacional influencie positivamente as dimensões, a participação familiar se mostrou influenciável quando observados trabalhos de educação nas fases inicias dos negócios tal qual este programa estudado. Esses resultados aprofundam a pesquisa empírica existente sobre o assunto, uma vez que a família se apresenta como uma bolha eudaimónica, financiamento os empreendimentos e pivotagem, apoiando quando necessário os futuros empreendedores.
O desenvolvimento de ferramentas auxiliares tais como o aplicativo Empreenda Agro Sustentável, alcançou o propósito, que foi de, servir como uma ferramenta para aprimorar os conteúdos ministrados sobre empreendedorismo sustentável no meio rural, fornecendo acesso a informações e ajudando a motivar o envolvimento com o auto-gerenciamento dos alunos participantes do programa. Esses objetivos foram atendidos adequadamente, oferecendo aos usuários participantes a oportunidade de revisar, reforçar e dispor de conteúdos direcionados.

O aplicativo também pode ser indicado para profissionais que tenham interesse em aprender mais sobre o desenvolvimento de negócios sustentáveis e escaláveis. O aplicativo já possui registro no Instituto Nacional de Propriedade Industrial INPI sob número de registro \textbf{BR 51 2019 002657 8} e disponível gratuitamente para testes na loja online \textit{Google Play Store}.

