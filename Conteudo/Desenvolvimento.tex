\chapter{METODOLOGIA}

%Metodologia da bibliografia
\section{Metodologia usada no Levantamento Bibliográfico}


Para atingir os objetivos que orientam este estudo, os procedimentos metodológicos foram planejados tendo como base programas educacionais que visam a promoção do empreendedorismo e comportamento empreendedor na condução dos cursos de graduação, em instituições de ensino públicas e particulares, como também Startups de natureza educacional. A metodologia de pesquisa utilizada encontra-se esquematizada na Figura \ref{figura_29}, os artigos foram exportados para o software StArt \cite{lapes_start_2016}. A ferramenta StArt foi desenvolvida para apoiar todo processo de Revisão Bibliográfica. Por meio de uma árvore hierárquica, categorizando os artigos em proximidade e níveis de aderência as palavras-chave \cite{hernandes_avaliacao_2010}. 

\begin{figure}[!htb]
\centering
\caption{\textbf{Planejamento da pesquisa e construção do portfólio de artigos}}
\includegraphics[scale=0.5]{Imagens/fases_pesquisa_bibliografica.png}
\fonte{Adaptado de \citeonline{hernandes_avaliacao_2010}}
\label{figura_29}
\end{figure}






\section{Programa Empreenda AGRO Sustentável: considerações metodológicas}


As atividades serão desenvolvidas em quatro encontros \textit{(workshops)}, que constam metodologias ativas, oficinas, palestras, e ferramentas tecnológicas visando à sinergia entre as estratégias de inovação no uso de tecnologias educacionais e os objetivos da proposta, com vistas a promover aprendizagem significativa e colaborativa. Todas as etapas do projeto de pesquisa serão executadas utilizando a metodologias ativas, objetivando o aprendizado do uso das ferramentas de gestão e desenvolvimento de negócios.
Durante os módulos do projeto (Workshops), os participantes testarão de seus projetos para que novas requisições sejam realizadas e/ou que erros nos planejamentos sejam encontrados e, consequentemente, debatidos e mitigados, utilizando para isso os métodos de modelagem de negócio \textit{(Lean Canvas e Business Model Canvas)}. Depois que todas as Sprints (atividades dos três Workshops) forem finalizadas, ou seja, que todos os módulos forem trabalhados, será iniciado um ciclo de Apresentações e desenvolvimento da habilidade de apresentação e demonstração dos produtos com apresentações (\textit{Pitch}). Destaca-se, ainda, que as inovações passiveis de registro intelectual apresentadas nesta pesquisa serão incentivadas a registro e documentação dos direitos.



\section{Planejamento Pedagógico}

O Programa Empreenda Agro Sustentável traz a proposta do trabalho ao ensino do empreendedorismo de forma multidimensional, multidisciplinar e disciplinado. Tal programa utilizou-se de Workshops particionados em temas que permitem a compreensão global do empreendedorismo, enquanto admite o ser humano como um ser multidimensional e culturalmente contextualizado no desenvolvimento de um negócio inovador. 

A proposta do programa é combinar oficinas que visam o desenvolvimento das características negociais e seus planos com palestras ministradas por profissionais de alto conhecimento nas áreas específicas de um negócio escalável. Com a aprendizagem baseada em equipes \textit{(team-based learning)} o programa delimitou inicialmente a inscrição dos participantes somente em equipes pré-estabelecidas, de modo a ter como iniciativa, maiores afinidades durante o desenvolvimento dos trabalhos. 


Com a finalidade de propor uma melhor mentoria no desenvolvimento dos objetivos propostos, o programa será desenvolvido em quatro encontros (Workshops) (Figura \ref{figura_17}), que abordarão temas pertentes ao empreendedorismo e o comportamento empreendedor, a saber:

\begin{itemize}

\item {1º Workshop \textbf{INSIGHT E DESPERTAR}: O que é startups, empreendedorismo, comportamento empreendedor e cultura empreendedora, problemas (segmentação do mercado), segundo os Objetivos do Desenvolvimento Sustentável (ODS), Modelagem do negócio e Criatividade;}
\item {2º Workshop \textbf{IDEAÇÃO}: A busca de oportunidades como característica empreendedora, construção do \textit{Lean Canvas}, mapa de empatia, validação da proposta de valor, economia colaborativa e \textit{coworking};}

\item {3º Workshop: \textbf{MAKEATHON}: Prototipagem para o MCVP, O que você pode fazer por seu cliente e como o cliente adquire seu produto?;}

\item {4º Workshop \textbf{DEMODAY}: Destaca-se que o mecanismo de pesquisa que será utilizado neste experimento foi composto por Cinco blocos de questões de múltipla escolha baseadas principalmente em escalas de cinco ou sete possibilidades.}
\end{itemize}

\begin{figure}[!htb]
\centering
\caption{\textbf{Jornada Empreenda Agro Sustentável}}
\includegraphics[scale=0.3]{Imagens/jornada.png}
\fonte{Próprio Autor}
\label{figura_17}
\end{figure}





\subsection{1º Workshop}

Buscando um engajamento maior dos participantes foram
aplicadas cinco estratégias/dinâmicas: Apresentação das ideias propostas inicialmente; Círculo Dourado;
Quem sou eu no Universo?; Descoberta do cliente usuário e o  desenvolvimento da proposta de valor Figura \ref{figura_30}.

\begin{figure}[!h]
\centering
\caption{\textbf{1º Workshop Empreenda Agro Sustentável}}
\includegraphics[scale=0.3]{Imagens/workshop-01.png}
\fonte{Próprio Autor}
\label{figura_30}
\end{figure}
%\newpage

A dinâmica \textbf{"Apresentação das ideias propostas"}\ visa a construção de um panorama visual de todos os insights, inscritas no programa, como também as temáticas mais buscadas pelos alunos. As atividades de Expressão Oral (EO) oportunizam aos estudantes a apropriação de recursos linguísticos e interativos inerentes às práticas orais e considerando que a EO pode ser uma eficaz ferramenta para a ensinagem de conteúdos conceituais, procedimentais e atitudinais em contato social \cite{baltar_genero_2010}.

A atividade proposta, considera que a expressão oral mesmo que incipiente proporcionara aos alunos e todo o grupo o exercício da criticidade, abordar e levantar seus pontos de vista e seus interesses na participação do programa pode executar a sua verve crítica, defender seus pontos de vista enquanto possibilita o aprimoramento das atitudes discursivas da ordem do expor e do argumentar, já que o estudante torna público suas expectativas e se compromete informalmente a comunidade de desenvolver suas propostas. 

Já a segunda dinâmica \textbf{Golden Circle} desenvolvido por \citeonline{sinek_golden_2015} tem por objetivo: criar ou desenvolver o valor de um novo produto, ideia ou negócio, tal dinâmica é pautada em três pilares: O quê, Como e Porquê. O Círculo Dourado e formado por três círculos de diferentes tamanhos que se completam. Ao centro está o \textbf{Porquê}, tal círculo objetiva a expressão do real propósito do negócio pensado pelo grupo. Este Refere-se ao conjunto de iniciativas e compromissos pensados para escalar e promover aos usuários e clientes o valor que a Startup realmente acredita. Resumidamente é o que a empresa de fato acredita ser o  diferencial perante a concorrência e ao mercado atual, este é o ponto de partida, principal círculo da dinâmica. 


\begin{figure}[!h]
\centering
\caption{\textbf{Golden Circle}}
\includegraphics[scale=0.3]{Imagens/circulo_dourado.png}
\fonte{Adaptado de: \cite{sinek_golden_2015}.}
\label{figura_5}
\end{figure}
\newpage

\subsection{2º Workshop}

Tendo a perspectiva macro do que venha a ser inovação e meios de negócios, o segundo encontro trará a proposta de descoberta de oportunidades e desenvolvimento do insight, por meio da busca de oportunidades e mineração de inovações Figura \ref{figura_31}.


\begin{figure}[!h]
\centering
\caption{\textbf{2º workshop Empreenda Agro Sustentável}}
\includegraphics[scale=0.3]{Imagens/workshop-02.png}
\fonte{Próprio Autor.}
\label{figura_31}
\end{figure}

Será construído o Mapa de empatia de modo a compreender os desejos e necessidades dos clientes e usuários até mesmo seu estado emocional, que influenciará os seus anseios a aquisição de produtos e serviços. Esta necessidade de entender e assimilar as necessidades dos usuários é prioridade no atendimento das expectativas e necessidades dos usuários, é um ponto fundamental para que as unidades de informação trabalhem orientadas à qualidade e com possibilidade de incorporação de serviços inovadores \cite{valdrich_mapa_2018}. O mapa de empatia é uma ferramenta visual para conduzir esta descoberta, ele é comporto por 6 (seis) reflexões diferentes sobre o cliente, são elas: \textbf{as dores}, \textbf{os ganhos},\textbf{o que ele escuta}, \textbf{o que ele vê}, \textbf{o que ele pensa e sente} e  \textbf{o que ele faz}, é possível entender o mapa de empatia por meio da figura \ref{figura_6}. 


\begin{figure}[h!]
\centering
\caption{\textbf{Exemplo do Mapa de empatia}}
\includegraphics[scale=0.4]{Imagens/mapa_empatia.jpg}
\fonte{\cite{osterwalder_value_2019}.}
\label{figura_6}
\end{figure}


Após o desenvolvimento da ideia da Startup por meio do mapa de empatia, é esperado que o participante tenha uma visão holística do que deseja produzir como negócio, assim para que seja visualizado as possibilidades e estruturas do negócio será trabalhado o quadro ferramenta Lean Canvas proposto por Ash Maurya, tendo como base para o desenvolvimento o \textit{Business Model Canvas} (BMC) entre outros materiais. Ele adaptou 4 quadros do BMC, buscando trabalhar aspectos de maior risco na criação de Startups \cite{maurya_running_2012}. Ele é ideal para o quando o negócio está no começo ou ainda não deu início às atividades e que ainda não fez testes sobre suas estruturas comerciais. Essa é a hora de analisar de maneira mais aprofundada os problemas que o mercado apresenta e estruturar de uma forma melhor a solução oferecida pela startup, encontrando aí o melhor resultado nessa equação \cite{sebrae_aprenda_2019}. O quadro é comporto pelos seguintes campos: \textbf{1.\textsuperscript{o} Problema, 2.\textsuperscript{o} Segmentos de Clientes, 3.\textsuperscript{o} Proposta Única de valor, 4.\textsuperscript{o} Solução, 5.\textsuperscript{o} Canais, 6.\textsuperscript{o} Fontes de Receitas e 7.\textsuperscript{o} Estrutura de Custos}. 

O problema deve ser as dificuldades que a startup deve resolver por seus usuários o segmento de Clientes: deve prever quais os clientes e usuários da sua 'startup'? De que forma eles podem ser segmentados? Na que torna o seu produto diferente e merecedor do dinheiro dos clientes, o que de fato a startup tem a oferecer ao cliente/usuário. No campo (Solução) o aluno deve explicar o menor conjunto de funcionalidades de seu produto, de que forma ele vai entregar a proposta de valor anteriormente pensada. Ao chegar nas métricas chaves eles deverão planejar como será a geração de capital deste negócio ou de que forma reterão seus clientes/usuários, no campo (Canais), será descrito uma lista de meios de marketing e captação de cliente. Para a estrutura de custos, os participantes descreverão todos os gastos sendo custos fixos e variáveis de seu negócio, já para o Fluxo de Receita ele identificará qual o tipo de modelo de receita – assinatura, anúncios, \textit{freemium}, e determine as premissas para indicadores como \textit{Life time value}, etc. Por fim, na vantagem competitiva deve ser explicado o algo que não pode ser comprado ou copiado, deve ser de fato a inovação de seu negócio, \cite{maurya_running_2012, sebrae_aprenda_2019}. É recomendável a construção do quadro seguindo esta sequência, na figura \ref{figura_7} é possível visualizar o quadro que será utilizado durante o programa. 



\begin{figure}[h!]
\centering
\caption{\textbf{Quadro Lean Canvas adaptado para o Programa Empreenda Agro Sustentável}}
\includegraphics[scale=0.2]{Imagens/canvas.png}
\fonte{Adaptado de: \cite{maurya_running_2012}.}
\label{figura_7}
\end{figure}
\newpage

\subsection{3º Workshop}
 
Neste Workshop será desenvolvido o Makeathon, momento no qual o aluno desenvolverá, de forma prática o seu Mínimo Produto Comercialmente Viável (MVBP sigla em inglês), que segundo \citeonline{aulet_empreendedorismo_2019} é um produto completo o bastante para que um cliente possa ganhar valor com ele, o qual deva possibilitar a demonstração concreta do que se pretende oferecer Figura \ref{figura_32}. 


\begin{figure}[h!]
\centering
\caption{\textbf{Makeathon Empreenda Agro Sustentável}}
\includegraphics[scale=0.4]{Imagens/workshop-03.png}
\fonte{Adaptado de: \cite{maurya_running_2012}.}
\label{figura_32}
\end{figure}



\subsection{4º Demoday}

E por fim o projeto terá como conclusão o dia de demonstração comercial chamado aqui de Demoday (Figura \ref{figura_33}), será o dia dedicado a demonstrar o quão inovador e interessante é o empreendimento em que foi desenvolvido pelas equipes durante todo o programa.


\begin{figure}[h!]
\centering
\caption{\textbf{Demoday Empreenda Agro Sustentável}}
\includegraphics[scale=0.4]{Imagens/workshop-04.png}
\fonte{Adaptado de: \cite{maurya_running_2012}.}
\label{figura_33}
\end{figure}



\section{Mobilização das equipes}

Inicialmente serão abertas vagas para estudantes do curso de graduação, ligadas aos cursos de Ciências Agrárias da Universidade Federal de Sergipe-UFS, onde as inscrições serão realizadas por equipe, desde que preencham os seguintes critérios:

	
\begin{itemize}
\item{Estarem regularmente matriculados e cursando quaisquer dos cursos e ao menos um aluno ligado as ciências agrárias;}
\item{Comprovarem disponibilidade de tempo para participação em todas as oficinas programadas;}
\item{Lidar com trabalhos em equipe.}
\end{itemize}

Objetiva-se o alcance de \textbf{120 alunos} participantes efetivos no programa, o qual será considerado o número de amostra para a pesquisa. Para delimitação das equipes foram elencadas 12 cadeias produtivas  prioritárias e áreas de desenvolvimento agrário destacado-se:

\begin{multicols}{2}
\centering
    \begin{itemize}
    \item{Agricultura Sustentável;}
    \item{Alimentos;}
    \item{Aplicações Molibe}
    \item{Automatização Agrícola}
    \item{Biotenologia}
    \item{Economia Criativa}
    \item{Fitoterapia}
    \item{Higiene/Produtos de beleza}
    \item{Turismo}
    \item{Pecuária Verde}
    \item{Tecnologias gerais}
\end{itemize}
\end{multicols}

\section{Questionário para análise de Campo}

A utilização de um método de pesquisa em uma dissertação depende da escolha de um modelo mais adequado ao problema da pesquisa e os objetivos pretendidos.

Assim, esta pesquisa será desenvolvida com caráter descritivo tendo como base o método de pesquisa \textit{Survey} descritivo, uma vez que este método busca contribui para o conhecimento geral de uma área particular de interesse, pois, envolve uma coleção de informações de indivíduos por meio de questionários e entrevistas sobre suas atividades ou sobre si mesmos \cite{forza_survey_2002}, como também diversos experimentos sobre o empreendedorismo na América Latina têm utilizado \textit{Surveys} realizados em residências ou com apelo direto aos donos de empresas para a coleta de dados, assim como em meios acadêmicos \cite{lima_ser_2015}. Na Figura \ref{figura_8} é possível compreender as fases da tendo como critério a pesquisa do tipo \textit{Survey}.

\begin{figure}[!htb]
\centering
\caption{\textbf{Fases da pesquisa tipo \textit{Survey}}}
\includegraphics[scale=0.4]{Imagens/survey.png}
\label{figura_8}
\fonte{Adaptado de \cite{moser_survey_2017}.}
\end{figure}

Desta forma, a pesquisa utilizará de um \textit{Survey} descritivo para analisar o potencial do comportamento empreendedor e a competências Empreendedoras, dos acadêmicos dos cursos de graduação em Ciências Agrárias inscritos no Programa Empreenda AGRO Sustentável, utilizando como fator indutor para melhoria o programa de extensão Empreenda Agro Sustentável a partir do modelo \textit{Global University Entrepreneurial Spirit Students’ Survey} (GUESSS), conhecido nacionalmente por Estudo GUESSS. Esta ferramenta de ensaio acadêmico que busca caracterizar o espírito, as atividades e as intenções empreendedores de estudantes universitários, de todos os níveis de aprendizagem e em todos os cursos universitários, bem como as condições de ensino e apoio a atividades empreendedoras. A pesquisa GUESS é realizada internacionalmente,  que em 2018 alcançou mais de 208.000 estudantes de mais de 3.000 universidades em 54 países inclusive no Brasil \cite{sieger_global_2018}.  Seu  principal  objetivo  é  acompanhar  indicadores perceptivos   de   variáveis   de   nível individual   e   contextual   do   ambiente   universitário, relacionados ao empreendedorismo entre estudantes de nível superior.

O questionário apresenta 5 conjuntos de questões dos mais diversos contextos relacionados ao empreendedorismo, medindo diferentes elementos do empreendedorismo e comportamento empreendedor no meio educacional tanto vindo dos discentes quanto dos docentes. O instrumento de pesquisa que será utilizado neste experimento foi composto por Cinco blocos de questões de múltipla escolha baseadas principalmente em escalas de cinco ou sete possibilidades. 

As perguntas iniciais têm por objetivo de traçar o perfil dos alunos entrevistados, tais como: gênero, faixa etária, curso vinculado e o perfil de interesse nas áreas de ensino ligadas ao empreendedorismo sustentável tal questionário foi baseado nos experimentos desenvolvido por \citeonline{lima_educacao_2014}. 


O Primeiro bloco contendo 11 questões abordará a ligação da família e o apoio familiar no empreendedorismo, tendo como alternativas, partido do “Discordo totalmente” a “Concordo totalmente”.

O Segundo conjunto e composto por 7 questões que analisam a intenção empreendedora do aluno da quais segue uma proporção partindo da resposta, tendo como alternativas, partido do “Discordo totalmente” a “Concordo totalmente”.

O Terceiro conjunto trata sobre os interesses dos alunos por disciplinas e atividades relacionadas ao empreendedorismo.

O Último conjunto é composto por 10 questões relacionadas a autoeficácia e a intenção em ter sua própria empresa ou ser autônomo composta por questões de múltipla escolha partindo da alternativa “Completamente Inseguro” a Completamente Seguro”. A figura \ref{figura_9} apresenta o universo das dimensões que serão conduzidas nesta pesquisa.


\begin{figure}[!htb]
\centering
\caption{\textbf{Modelo sobre universo das dimensões estudadas nesta pesquisa sobre a intenção empreendedora}}
\includegraphics[scale=0.3]{Imagens/intencao_empreendedora.png}
\fonte{Adaptado de \cite{vale_motivacoes_2014}.}
\label{figura_9}
\end{figure}
\newpage

Desta forma a pesquisa caracteriza-se como uma pesquisa de levantamento \textit{Survey} do tipo Descritivo sob corte Longitudinal, já que \citeonline{tormen_potencial_2005} descreve que este tipo de instrução se destaca por compreender uma amostra expressiva em relação ao universo pesquisado. 

\section{Universo da Pesquisa}

O universo desta pesquisa é composto por \textbf{1.143 discentes} ativos dos cursos de graduação nas áreas de agrárias da Universidade Federal de Sergipe (UFS): Engenharia Agronômica, Engenharia Agrícola, Zootecnia, Engenharia Florestal, Medicina Veterinária e Engenharia de Pesca, dados contidos no relatório estatístico de matrículas 2019 da instituição, \cite{campelo_ufs_2018}. É importante considerar que esta pesquisa não considerou classe social local de ensino anterior, e desempenho acadêmico do aluno durante a graduação.

\section{Análises Estatísticas}


Buscando quantificar a homogeneidade das questões, foi utilizado uma análise fatorial e análise de variância multivariada buscando aglutinar as variáveis de cada questão em fatores. Tais fatores forão preparados com o uso da a análise de fatores ortogonais, com rotação Varimax, por meio do método de esfericidade Bartlett e KMO com o nível de significância p < 0,05, este ensaio tem o objetivo de aglutinar em fatores únicos os dados obtidos com diferentes itens de escala do questionário \cite{hair_multivariate_2006}. Esta pesquisa levará em consideração quatro conjuntos de variáveis aglutinados que influenciam na natureza da resposta empreendedora, lembrando que diversos autores abordam diferentes pontos sobre esta temática, visto que este programa se comporta como uma fase de pre aceleração, abordaremos os fatores que podem ter comportamento de conjunto de variáveis independentes, quando submetidas à participação dos alunos no programa, a Figura \ref{figura_9} representa as 4 dimensões estudadas:


\begin{itemize}
\item {Interesse em conteúdos relacionados a educação empreendedora;}
\item {Dimensão da Autoeficácia dos estudantes;}
\item {Dimensão da intenção pretensão ao empreendedorismo dos estudantes;}
\item {Dimensão da participação familiar e influência de terceiros no desenvolvimento empreendedorismo;}
\end{itemize}

Buscando analisar a dimensão \textbf{Interesse em conteúdos relacionados a educação empreendedora;} após a participação no programa, será utilizado uma Análise de Variância Multivariada (MANOVA) caso os dados satisfaçam as exigências para tal teste. A MANOVA é uma técnica estatística que analisa independentemente grupos distintos de amostras buscando avaliar as diferenças entre as médias por grupos. A MANOVA tem por objetivo verificar diferenças de grupos de variáveis categóricas (independentes) quanto aos seus impactos sobre diversas variáveis métricas (dependentes) em paralelo  \cite{hair_alise_2009}. 

Este modelo aplica-se a esta pesquisa, pois ela tem por objetivo avaliar as alterações quanto ao perfil empreendedor em grupos de alunos participantes do Programa de extensão Empreenda Agro Sustentável. Desse modo, é possível avaliar se as diferenças entre os níveis médios dos grupos captados pela \textit{}{survey} são significativas entre os grupos e dentro dos grupos \cite{rocha_avaliacao_2014}. 

A hipótese nula aqui apresentada será utilizada para condução deste experimento. Com a rejeição desta, a hipótese alternativa é comprovada \ \cite{hair_alise_2009}.

\textbf{H0:} Não há diferença entre as médias das medidas que averíguam o perfil empreendedor entre os alunos participantes do Programa de extensão Empreenda Agro Sustentável.

\textbf{H0} Não há diferença entre as médias das medidas que averíguam o perfil empreendedor entre os alunos participantes do Programa de extensão Empreenda Agro Sustentável.



Para o processamento dos dados será utilizado o Software IBM SPSS \cite{ibm_corp_ibm_2017}. 


\section{Tamanho da amostra}

Muitas  vezes  não  se  faz necessário,  ou  é não é possível,  dispor  de toda a população objetivo  do  projeto. Desta forma se faz necessário dispor de uma parte do universo da pesquisa para que seja possível realizar inferências confiáveis da população total \cite{marino_manual_2003}.

Para que seja possível a análise populacional de forma fidedigna, e selecionada uma amostra de tal população. Amostra é o número de pessoas a que serão entrevistadas nesta pesquisa. Esse número tido inicialmente como o número máximo de capacidade de condução efetiva do programa, partindo do universo pesquisado (CCAA) ele representa 10,5\% dos alunos ativos no centro, número satisfatório para avaliação de projetos sociais segundo \citeonline{marino_manual_2003}, segundo o autor quando o número total do grupo próximo a 1.000 é sugerido uma amostra de 50 a 100 participantes. 

A amostra desta pesquisa compreenderá \textbf{120 discentes} que participarem do Programa Empreenda Agro Sustentável que responderão o questionário instrumento de pesquisa que serão aplicados durante os Workshops. Tais Workshops ocorrerão nos meses de agosto, outubro e dezembro de 2019. Tal instrumento será aplicado presencialmente. 


\section{Considerações Éticas}

Por critérios éticos precedendo ao início do questionário, foi inserido um Termo de Consentimento Livre e Esclarecido (\textbf{TCLE}), composto por esclarecimentos sobre a pesquisa, além da solicitação de autorização para o uso dos dados e por ventura imagem que seja necessária ao desenvolvimento do experimento. questionário aplicado nesta pesquisa atende os termos das Resoluções n. 466 de 12 de dezembro de 2012 do Conselho Nacional de Saúde \cite{cns_resolucao_2012}, o qual por se tratar de pesquisa com seres humanos foi submetido ao Comitê de Ética em Pesquisa Envolvendo Seres Humanos (\textbf{CEP}) e a Comissão Nacional de Ética em Pesquisa (\textbf{Conep}) por meio da plataforma Brasil Saúde sendo \textbf{APROVADO} sob o número do Certificado de Apresentação para Apreciação Ética \textbf{CAAE: 23853219.4.0000.5546}.




