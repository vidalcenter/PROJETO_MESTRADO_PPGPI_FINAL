\chapter{REFERENCIAL TEÓRICO}

\section{Desenvolvimento Rural Sustentável}

Definir o desenvolvimento rural sustentável requer um esforço observacional e prático, pois, este ambiente vem sofrendo profundas transformações em suas demandas e necessidades. O desenvolvimento que antes se apresentava majoritariamente como produção de subsistência, hoje dá lugar a um complexo sistema agroindustrial \cite{bastos_determinantes_2018} e social. É importante neste sentindo compreender que definir o desenvolvimento rural com apenas um conceito seria uma proposição simplista do contexto de desenvolvimento rural. Partindo da definição de consequência de ações governamentais definidas por \citeonline{navarro_desenvolvimento_2001} como "ações práticas", este autor descreve que o:

\begin{citacao}
“[...] Desenvolvimento rural, portanto, pode ser analisado a posteriori, neste caso se referindo às análises sobre programas já realizados pelo Estado (em seus diferentes níveis) visando a alterar facetas do mundo rural a partir de objetivos previamente definidos. Mas pode se referir também à elaboração de uma "ação prática".
\end{citacao}

O desenvolvimento rural também pode ser compreendido por meio de um conceito mais regional definido como: "Desenvolvimento Local". Tal expressão é recente e deriva de iniciativas de mobilização organização social no sentido de promover uma maior representação dos diferentes atores sociais no processo de desenvolvimento. E que o Estado assume papel de agente facilitador desse processo de descentralização das políticas públicas para ser democrático, buscar transparência de suas instituições, o equilíbrio das forças exercidas pelas diferentes correntes de interesse e o compromisso com a qualidade de vida na população afetada  \cite{campanhola_diretrizes_2000}. 
Tal conceito demonstra o espaço rural como um local ideal para a promoção de políticas de inovação e a construção de padrões inovadores na relação entre populações e instâncias públicas, numa tentativa de rompimento com a dominação, que parte de baixo para cima. Neste contexto, surge as Organizações Não Governamentais (ONGs) que buscam garantir a participação da população local, e fazer valer tais mudanças atuando normalmente em ambientes geograficamente mais restritos (região rural, povoados ou municípios),  \cite{assis_agricultura_2005, campanhola_diretrizes_2000}.

Este trabalho está direcionado em estudos relacionados ao Desenvolvimento Rural Sustentável. Anteriormente, o conceito de Desenvolvimento Rural Sustentável era denominado por "Progresso Rural", pois, havia um entendido genérico como sentido parcial e prático de “melhoramento do ambiente” \cite{almeida_da_1995}. Entretanto, torna-se imprescindível destacar que, o desenvolvimento sustentável no meio rural não pode ter suas bases de compreensão apenas no progresso econômico, local ou regional. 
Se mostra de suma importância entender que para compreender a sustentabilidade é necessário ter um olhar sistêmico que permeie todo o processo, envolvendo diversas dimensões, dentre as quais se destacam a econômica, a sociocultural, a político-institucional e a ambiental \cite{vieira_politica_2015}, a ação de desenvolvimento sustentável é por um lado fruto do desenvolvimento social, por outro lado, esta ação contribui com o desenvolvimento da sociedade de forma autossustentável, ao introduzir inovações anti-predatórias, ao satisfazer demandas específicas tendo como base a economia circular e ao tornar mais densas as redes de cooperação buscando a autossuficiência consciente, satisfazendo as necessidades no presente, sem comprometer a capacidade das gerações futuras de suprir suas próprias necessidades \cite{onu_sustainable_2016}.


O rural deve ser visto segundo \cite{kageyama_desenvolvimento_2008} como, uma amálgama de práticas heterogêneas, estilos mutuamente contrastantes, tendências de desenvolvimento divergentes, posições hegemônicas e mudanças quase subterrâneas que, a princípio, são praticamente imperceptíveis, mas que, por fim, podem mudar todo o sistema de produção. Compreender a complexidade do rural se faz necessário uma vez que a simples padronização do ambiente é um conceito reducionista do campo, uma maneira concisa do que ocorre no rural \cite{van_der_ploeg_trajetorias_2011}. 

O empreendedorismo é uma das ferramentas possíveis para promoção de desenvolvimento do campo e que considera suas complexidades,  \citeonline{autio_retaining_2016} afirmam que um euro de financiamento público para as iniciativas em empreendedorismo gerou 1,11 euro de crescimento das vendas excedentes. Para que seja aplicado corretamente, se faz necessário compreender melhor o empreendedorismo sustentável como também a aplicação prática no meio rural. 

Segundo \citeonline{dornelas_como_2003}, o empreendedorismo significa fazer algo novo, diferente, mudar a situação atual e buscar, de forma incessante, novas possibilidades de negociações, tendo como foco a inovação e a criação de valor, outrossim, \citeonline{leite_aprendizagem_2015} trata o empreendedorismo como  um  processo,  que se concentra em iniciar  e  gerir  empreendimentos,  isto  é,  o conjunto  de  conceitos,  métodos,  instrumentos  e  práticas  relacionadas  com  a criação, implantação  e  gerenciamento de novas  empresas  ou organizações.

Existem diversas definições de empreendedorismo, mas a essência resume-se na inovação, ou seja, criação de algo novo ou modificação de algo buscando uma nova aplicação. Empreender é empregar os recursos disponíveis de forma criativa, assumindo riscos calculados e buscando oportunidades, ou seja, é um processo de criação de um negócio de valor com recursos limitados tornando-o capitalizável e economicamente viável  \cite{costa_empreendedorismo_2006, stevenson_new_1989, lopes_educacao_2010}. Apesar dessa diversidade conceitual, a ideia de empreendedorismo tem sido predominantemente associada às concepções de progresso e tecnologia usual deixando de lado o campo e nele suas aplicações práticas.  

O intenso debate sobre desenvolvimento da agricultura brasileira de forma sustentável em consonância com assuntos econômicos de interesse nacional, torna o tema desta pesquisa oportuno e atual, haja vista que o que se produz na agricultura no Brasil correspondeu a 19\% do total das exportações no ano de 2018 \cite{mdic_comex_2019}. Entretanto, este meio de produção convive com a limitação dos recursos naturais \cite{jacobi_meio_1999}, levando ao Estado pensar em políticas públicas que busquem soluções para as demandas tecnológicas surgidas no meio rural, e gerar profissionais capazes de compreender a complexidade da intensa produção no ccampo, mantendo o ritmo constante das mudanças tecnológicas ao mesmo tempo em que convivemos com o uso de limitados recursos naturais  \cite{costa_dinamica_2016}.

No alcance desse modelo sustentável, um profissional empreendedor deve ser preparado desde a academia por meio da educação empreendedora de modo que seja capaz de melhorar o desempenho produtivo \cite{da_silva_qualidade_2017}, a capacidade competitiva, a melhoria da segurança alimentar do país \cite{hoffmann_brasil_2015} e, ao mesmo tempo garantir a perpetuação da manutenção do meio ambiente, e não apenas replicar novos padrões de produção e distribuição de bens e serviços e do uso dos recursos naturais, além disso, este profissional deve ser capaz de inovar \cite{morais_empreendedorismo_2018}.



\section{Agritechs}

Atualmente passamos por uma fase muito expressiva da disseminação do empreendedorismo no Brasil e no mundo, tendo como exemplo o crescimento das "Startups".A velocidade do desenvolvimento e conexões das negociações antes realizadas de pessoa a pessoa, atualmente passa pelo campo da automação e digitalização aumentando a velocidade da inovação interferindo na relação entre pessoas e produtividade  \cite{campos_o_2016}. 

Independentemente do tamanho e demanda que venha a ter uma atividade comercial, o desempenho negocial do empreendimento depende da capacidade de adaptação e resiliência do administrador, assim também são os negócios no meio rural. O grande produtor objetivando o desenvolvimento de capital utiliza-se de um vasto corpo de recursos humanos e tecnológicos. No entanto, os pequenos produtores muitas vezes dependem apenas deles mesmos, sendo proprietário e administrador, ou quando exige a possibilidade de contratação passa a depender de apenas um profissional, responsável por lidar com todos os entraves da produção agrícola e das mudanças constantes do meio rural \cite{soares_relacao_2017}. 

Uma das alternativas possíveis para reduzir os riscos e se manter ativo e produtivo na atividade profissional, é o investimento em tecnologia e inovação tais como: sementes melhoradas, adubação agrícola mais eficiente, em centros coletivos de pesquisa direcionadas ao campo (Universidades e empresas) \cite{bochi_dorneles_coletivos_2014, gomes_inovacao_2014} e pequenas empresas prestadoras (startups) de serviços ou nichos do mercado rural. Estes negócios objetivam a melhoria de determinada área agropecuária \cite{junior_agtechs:_2019} ou necessidade do negócio, de forma efetiva e economicamente viável. 

As startups surgiram por meio das oportunidades que os negócios e as necessidades lhes apresentaram, transformando as necessidades e ideias em negócio viável e sustentável. As startups se ajustam muito bem ao meio rural já que para obter sucesso e alcançar o crescimento rápido na produtividade agrícola é necessária uma capacidade de gerar tecnologias adaptativas e ecológicas \cite{contini_hayami_2019}.

O modo como se processa a diversificação tecnológica no campo relaciona-se diretamente com o desenvolvimento e a adaptação de novas tecnologias agrícolas e a diversidade das condições socioeconômicas e ambientais \cite{fen-azmeyer_o_2019}. Neste ambiente de cocriação surgem as Startups direcionadas à agricultura. Tais empresas de base tecnológica são focadas em soluções para o agronegócio e muitas vezes são referenciadas como um setor \textit{Agtech} \cite{blanco_agtechs:_2019}.

Dentro do espectro de novos empreendimentos, os mais comuns para o agronegócio no Brasil são: \textit{Business to Business} (B2B), \textit{Business to Consumer} (B2C), \textit{Business to Business to Consumer} (B2B2) e o \textit{Direct to Consumer} (D2C). Segundo \citeonline{junior_agtechs:_2019} e \cite{abstartups_startupbase_2019}, as principais  áreas de atuação destas Startups são as áreas de: Biotecnologia de Alimentos inovadores atendendo tanto B2B quanto B2B2C, Marketplace do agronegócio (B2B2C), Bioenergia e Biomateriais (B2B), indústria de Software como Serviço SaaS (B2B).


\section{Comportamento Empreendedor e Educação empreendedora}


A Intenção empreendedora (IE) é a matriz de toda atividade e desenvolvimento ao empreendedorismo e surgimento de novos negócios, esta intenção pode ser vista como o primeiro passo no processo empreendedor \cite{zhao_relationship_2010, shirokova_exploring_2016}. Estudos recentes demonstram que a IE para abertura de negócios vai muito além do dualismo oportunidade-necessidade, ou seja, a criação e/ou descoberta de oportunidades, faz parte também o medo do desemprego, e a  incapacidade de adaptação as mudanças técnicas e tecnológicas especialmente em países em desenvolvimento \cite{vale_motivacoes_2014}. As motivações extrapolam a lógica binária oportunidade/necessidade, e agrupam-se em seis componentes: identificação  de  oportunidade;  atributos/expectativas pessoais; ambiente  externo em particular associado ao mercado de trabalho; influência  de  terceiros, insatisfação com emprego; influência familiar, \cite{vale_motivacoes_2014, rodrigues_intencao_2019,ferreira_intencao_2017}.

Buscando lidar com tais variáveis, conceitos e ferramentas psicológicas devem ser aplicados não apenas a ambientes empresariais, mas durante toda a formação acadêmica, em combinação com um profundo conhecimento de pesquisa e negócios \cite{zhao_relationship_2010} A existência do empreendedorismo reside em tomadas de decisões inovadoras, das quais não se separam das características intrínsecas do individuo e suas experiências durante sua construção pessoal, por meio de trocas sociais \textit{networking} \cite{de_souza_alencar_intencao_2019}.

A inovação, a propagação da inovação e o surgimento de novos empreendimentos, em muitos países, são tidos como importantes sinais para o crescimento e recuperação de crises econômicas \cite{silva_mudancestrutural_2017}. Tais sinais de desenvolvimento estão ligados diretamente ao desenvolvimento intelectual do capital humano. Segundo o \citeonline{reis_capital_2017} o capital humano é o insumo fundamental da Pesquisa e Desenvolvimento (P\&D), e a P\&D é a condição \textit{sine qua non} para a geração e a intensidade de novidades, e as inovações catalisam e dinamizam o processo de crescimento econômico.

O Investimento no  capital  humano desde a formação acadêmica permite também o surgimento de melhorias no ambiente laboral e aumenta os níveis de produtividade e renda dos futuros profissionais \cite{macedo_capital_2019}. Até pouco tempo, os currículos educacionais nas escolas e cursos relacionados a administração no Brasil focavam quase que totalmente ao atendimento às necessidades do mundo corporativo, deixando de lado o fator (criatividade e inovação). 

Buscavam profissionais técnicos na prevenção de riscos ao invés de formar líderes criativos que criem estratégias de prevenção, assumam riscos \cite{sanna_evolution_1999} e os solucione, além da manipulação do ambiente externo, gerenciar o crescimento e a inovação em pequenas e médias empresas exige que os novos empreendedores possuam uma capacidade distinta de tomar e implementar decisões e fortes habilidades de liderança \cite{palmer_chip_2019}. 

Parece, deste modo, interessante investigar a figura do aluno como sujeito potencialmente empreendedor, como uma pessoa capaz de identificar oportunidades, criar negócios, e pode reunir os recursos necessários face ao risco e incerteza, \cite{pietrovski_alise_2019}.


O mercado econômico emergente, as necessidades de entregas urgentes e a redução cada vez maior das ofertas de emprego levou os centros de ensino a iniciarem o desenvolvimento deste conteúdo disciplinar e os demais conteúdos relacionados. No início dos anos 80 o empreendedorismo estava diretamente ligado ao desenvolvimento econômico e à criação de postos de trabalho em um país \cite{rodrigues_intencao_2019}, passando a ser visto como importante fator a ser explorado nas comunidades acadêmicas. 

É nesse contexto que surge em 1981, o ensino do empreendedorismo no Brasil tendo como precursor o Professor Ronald Degen \cite{degen_o_1989} na Escola de Administração de Empresas de São Paulo (EAESP) pertencente a Fundação Getúlio Vargas (FGV),Na literatura voltada para o tema, destaca-se as pesquisas de: \citeonline{dolabela_oficina_1999}; \citeonline{duarte_sesi_2004}; \citeonline{pires_empreendedorismo_2006}; \citeonline{ramos_o_2005}, \citeonline{branca_terra_o_2006}, entre outros. A Tabela \ref{tabela_1} desenvolve um recorte histórico do ensino da área no Brasil, tendo como corte temporal 1980 a 2007, segundo \cite{fernandes_breve_2013}. 



\begin{longtable}{lp{11cm}}

\caption{\textbf{Histórico do Ensino de Empreendedorismo no Brasil}}\label{tabela_1} \\ \hline \hline


\hline \multicolumn{1}{p{2cm}}{\textbf{Ano}} & \multicolumn{1}{p{11cm}}{\textbf{Ocorrência}}\\ \hline 

\endfirsthead


\multicolumn{2}{c}%

{{\bfseries \tabname \ \thetable{} -\ \textbf{Continuação}}}\\

\hline \multicolumn{1}{p{2cm}}{\textbf{Ano}} & \multicolumn{1}{c}{\textbf{Ocorrência}}  \\ \hline 

\endhead

\hline \multicolumn{2}{r}{{\textbf{Continua}}} \\ \hline

\endfoot
\hline \multicolumn{2}{r}{{\textbf{Continua}}} \\ \hline

\endfoot
\hline \multicolumn{2}{r}{{\textbf{Conclusão}}} \\ \hline
\hline \hline

\endlastfoot


1980 & Fundação Getúlio Vargas - FGV implanta o ensino formal de empreendedorismo no Brasil;  \\\\\hline
1980 & A Universidade de São Paulo - USP institui polo de ensino de empreendedorismo;  \\\\\hline
1981 & O Professor Ronald Degen leciona a disciplina “Criação de Negócios”  \\\\ \hline
1984 & O Professor Sílvio dos Santos, da USP leciona disciplina referente à criação de novas \\\\ \hline
1991 & A Professora Ofélia Sette Torres funda o Centro de Empreendedorismo;  \\\\\hline
1991 & É introduzido no Brasil o Programa Empretec, da Organização das Nações Unidas - ONU, para
capacitar empreendedores;  \\\\\hline
1993 & O Empretec passa a ser coordenado pelo Sebrae no Brasil;  \\\\ \hline
1996 & O Professor Paulo Goldsmith, coordena a versão brasileira da competição internacional Global Moot Corp, realizada pela Universidade do Texas desde 1984. Em 2001 essa competição foi aberta a todas as escolas da América Latina, passando a se chamar Latin America Moot Corp;\\\\ \hline
1999 & Lançamento do livro “O Segredo de Luísa”, do Professor Fernando Dolabela, renomado
especialista em educação empreendedora no Brasil e criador da Pedagogia Empreendedora;  \\\\ \hline
2000 & Os professores Tales Andreassi e Marcelo Aidar passam a ministrar curso de empreendedorismo
na EAESP;  \\\\ \hline
2002 & O Professor José Antônio Lerosa de Siqueira funda na USP o Centro Minerva de Empreendedorismo;  \\\\\hline 
 2005 & É realizada a primeira Semana do Empreendedorismo pelo Centro de Empreendedorismo e Novos Negócios da FGV. Atualmente o centro é o responsável pela Latin America Moot Corp e pela competição Sumaq16 de empreendedorismo social;  \\\\ \hline 
 2007 & A FGV é pioneira ao estabelecer como obrigatórias disciplinas que tratem do tema empreendedorismo nas grades curriculares dos cursos de graduação em administração pública e de empresas da EAESP; \\\\ \hline

\end{longtable}
 \fonte{\cite{almeida_aprendizagem_2019}}


Fica nítido que a preocupação com o ensino de empreendedorismo está saindo de sua fase embrionária e se consolidando nos principais centros de graduação e pós-graduação, nos mais diversos segmentos de formação acadêmica, desde cursos de engenharia, passando por desenho industrial, e até mesmo turismo \cite{henrique_praticas_2008}. A educação empreendedora como investimento ao capital humano, a além de fortalecer a criação de produtos e a dinamização de atividades econômicas, torna-se uma possibilidade de combater o desemprego \cite{morais_empreendedorismo_2018} e a redução das jornadas de trabalho e custos com materiais. Para \citeonline{schaefer_formacao_2017}, o indivíduo empreendedor é o ator capaz de inovar no processo evolutivo do mundo contemporâneo, além de ser capaz de resolver problemas e absorver oportunidades e capaz de lidar com as constantes inversões do mercado econômico. Desta forma se faz necessário um robusto incentivo ao empreendedorismo e a promoção de sua cultura visto que, a criação de negócios está diretamente ligada à ação empreendedora, processo dinâmico que possibilita o desenvolvimento de empregos e riquezas, impactando na prosperidade de diversas regiões do Brasil \cite{leite_aprendizagem_2015}.
No Brasil existe uma progressiva necessidade do ensino do empreendedorismo correto e escalável, sendo um país que segue um constante crescimento de empregos informais Esse é o país que apresentou no ano de 2019 um contingente de pessoas que conseguiram trabalho em condição de informalidade, que atingiu um recorde da série histórica, iniciada em 2012, chegando a 41,4\% de recursos humanos ocupados no Brasil \cite{ibge_informalidade_2019}. Foi também observada uma taxa crescente de desocupação nas idades iniciais de empregabilidade (14 aos 24 anos) desde 2012, segundo dados obtidos na Pesquisa Nacional por Amostra de Domicílios Contínua - PNAD Contínua  \cite{ibge_informalidade_2019}, como também uma taxa crescente de desocupação nas idades iniciais de empregabilidade (14 aos 24 anos) desde 2012, dados obtidos na Pesquisa Nacional por Amostra de Domicílios Contínua - PNAD Contínua \cite{ibge_instituto_brasileiro_de_geografia_e_estatistica_pesquisa_2019},(Figura \ref{figura_2}). É sabido que os alunos graduandos ainda apresentam lacunas de formação em seu potencial empreendedor e que cabe às universidades criar processos de ensino e aprendizagem que preencham esses espaços  \cite{pietrovski_alise_2019}.


\begin{figure}[!htb]
\centering
\caption{\textbf{Taxa de desocupação por idade, 1º trimestre 2012 - 3º trimestre 2019}}
\includegraphics[scale=0.25]{Imagens/taxa_desocupacao.png}
\fonte{\citeonline{ibge_instituto_brasileiro_de_geografia_e_estatistica_pesquisa_2019}}
\label{figura_2}
\end{figure}
\newpage

Com as universidades e institutos de ensino superior sendo reconhecidamente contextualizados como promotores da inovação no Brasil, país que configura o 13.º lugar entre os maiores produtores de publicações de pesquisa (\textit{papers}) e inovação em nível mundial, \citeonline{clarivate_analytics_web_of_research_2017}. 
Logo, as universidades brasileiras precisam se ajustar a esse novo paradigma educacional, que situa as instituições de ensino superior no campo da promoção do empreendedorismo direcional e sistemático, assim como o comportamento empreendedor. A educação empreendedora disciplinada mostra-se eficaz no tocante ao surgimento das inovações, direcionadas e a promoção da identidade empreendedora para novos negócios \cite{jain_academics_2009} uma vez que a universidade vem a ser um local privilegiado do saber, da liberdade acadêmica e da experimentação científica, e tem a prerrogativa de poder tornar o empreendedorismo como um conteúdo de conhecimento, e uma ferramenta capaz de gerar inovações \cite{dolabela_oficina_2008}. 


Atualmente, diversos tipos de métodos e ferramentas de ensino podem ser utilizados para auxiliar o docente e, consequentemente, motivar os estudantes em sua experiência na aprendizagem. Para atingir os objetivos da educação empreendedora, é preciso promover reflexões nos campos do ensino, formação de professores, uso dos recursos e infraestrutura \cite{marques_experiencia_2019}. O desenvolvimento do interesse ao empreendedorismo envolve diversos conteúdos de aprendizado, e é necessário organizar as metodologias e suas aplicações pedagógicas \cite{rocha_avaliacao_2014}. O mesmo autor elencou os Principais Métodos, Técnicas e Recursos Pedagógicos no Ensino do Empreendedorismo (Tabela \ref{tabela_2}). 



\begin{longtable}{p{3.5cm}p{11.0cm}}

\caption[\textbf{Principais  Métodos, Técnicas e Recursos Pedagógicos no Ensino de Empreendedorismo}]{\textbf{Principais  Métodos e Recursos Pedagógicos no Ensino de Empreendedorismo}} 
\label{tabela_2} \\


\hline \hline \multicolumn{1}{p{3.5cm}}{\textbf{Métodos, Técnicas e Recursos}} & \multicolumn{1}{c}{\textbf{Aplicações}}\\ \hline 

\endfirsthead


\multicolumn{2}{c}%

{{ \bfseries \tablename \ \thetable{} - \ \textbf{Continuação}}}\\

\hline \multicolumn{1}{p{3.5cm}}{\textbf{Métodos, Técnicas e Recursos}} & \multicolumn{1}{c}{\textbf{Aplicações}}  \\ \hline 

\endhead

\hline \multicolumn{2}{r}{{\textbf{Continua}}} \\ \hline

\endfoot
\hline \multicolumn{2}{r}{{\textbf{Continua}}} \\ \hline

\endfoot
\hline \multicolumn{2}{r}{{\textbf{Conclusão}}} \\ \hline
\hline \hline

\endlastfoot

Aulas expositivas & Transferir conhecimentos sobre o Empreendedorismo, as características pessoais do empreendedor, os processos de inovação, fontes de recursos, financiamentos e aspectos legais de pequenas empresas.  \\

Visitas e contatos com empresas & Estimular o \textit{network} e incitar o estudante a sair dos limites da IES para entender o funcionamento de mercado na vida real. Desenvolver visão de mercado.  \\

Plano de negócios & Desenvolver as habilidades de planejamento, estratégia, marketing, contabilidade, recursos humanos, comercialização. Desenvolver a habilidade de avaliação do novo negócio, analisando o impacto da inovação
no novo produto ou serviço. Construir habi\lidade de avaliar e dimensionar riscos do negócio pretendido. \\ 

Estudos de situações problemas & Construção da habilidade de pensamento crítico e de avaliação de cenários e
negócios. Desenvolver a habilidade de interpretação e definição de contextos associados ao Empreendedorismo. \\ 

Trabalhos teóricos em grupo & Construção da habilidade de aprender coletivamente. Desenvolver a
habilidade de pesquisar, dialogar, integrar e construir conhecimentos,
buscar soluções e emitir juízos de valor na realização do documento escrito. \\ 

Trabalhos práticos em grupo & Construção da habilidade de atuar em equipe. Desenvolver a habilidade de planejar, dividir e executar tarefas em grupo, de passar e receber críticas construtivas. Ampliar a integração entre o saber e o fazer.  \\ 

Grupos de discussão & Desenvolver a habilidade de testar novas ideias. Desenvolver a capacidade de avaliar mudanças e prospectá-las como fonte de oportunidades. \\ 
 
\textit{Brainstorming}  & Construção da habilidade de concepção de ideias, prospecção de
oportunidades, reconhecendo-as como oportunidades empreendedoras. \\ 


Seminários e palestras com empreendedores & Transferir conhecimentos das experiências vividas por empreendedores
desde a percepção e criação do produto, abertura do negócio, sucessos e
fracassos ocorridos na trajetória empreendedora. \\ 

Criação de empresa & Transpor as informações do plano de negócios e estruturar os contextos necessários para a formalização. Compreender várias etapas da evolução da empresa. Desenvolver a habilidade de organização e planejamento operacional. \\ 

Aplicação de provas dissertativas & Testar os conhecimentos teóricos dos estudantes e sua habilidade de
comunicação escrita. \\ 

Atendimento individualizado & Desenvolver a habilidade de comunicação, interpretação, iniciativa e
resolubilidade. Aproximar o estudante do cotidiano real vivido nos pequenos negócios. \\ 

Trabalhos teóricos individuais & Construção da habilidade de concepção de conhecimento individualizado,
estimulando a autoaprendizagem. Induzir o processo de autoaprendizagem. \\ 

Trabalhos práticos individuais & Construção da habilidade da aplicação dos conhecimentos teóricos
individuais, estimulando a autoaprendizagem. Estimular a capacidade
laboral e de auto realização. \\ 

Criação de produto & Desenvolver habilidade de criatividade, persistência, inovação e senso de
avaliação. \\ 

Filmes e vídeos & Desenvolver a habilidade do pensamento crítico e analítico, associando o
contexto assistido com o conhecimento teórico. Estimular a discussão em grupo e o debate de ideias. \\ 

Jogos de empresas e simulações & Desenvolver a habilidade de criar estratégias de negócios, solucionar
problemas, trabalhar e tomar decisões sob pressão. Aprender pelos próprios erros. Desenvolver tolerância ao risco, pensamento analítico, comunicação intra e intergrupais. \\ 

Sugestão de leituras & Prover ao estudante teoria e conceitos sobre o Empreendedorismo. Aumentar a conscientização do ato empreendedor. \\ 
Incubadoras & Proporcionar ao estudante espaço de motivação e criação da nova empresa, desenvolvendo múltiplas competências, tais como habilidades de liderança, organizacionais, tomada de decisão e compreender as etapas do ciclo de vida das empresas. Estimular o fortalecimento da network com financiadores, fornecedores e clientes. \\

Competição de planos de negócios & Desenvolver habilidades de comunicação, persuasão e estratégia.
Desenvolver capacidade de observação, percepção e aplicação de melhorias no padrão de qualidade dos planos apresentados. Estimular a abertura de empresas mediante os planos vencedores. \\ 

\end{longtable}
\fonte{\cite{rocha_avaliacao_2014}}


Os centros de ensino devem contribuir para o desenvolvimento da “cultura empreendedora” por meio da “educação empreendedora ativa” \cite{tscha_empreendendo_2014}, que incentive tanto docentes quanto aos discentes a “despertarem dentro de si o espírito empreendedor e a explorar o espaço potencial para o empreendedorismo, transformando realidades por meio dos empreendimentos que podem desenvolver economicamente e socialmente um país e uma sociedade” \cite{tscha_empreendendo_2014}. Uma vez que, a cultura empreendedora depende diversos fatores influenciáveis ao longo do aprendizado e vida. \cite{dornelas_empreendedorismo_2005} explica que  empreender segue fluxo um definido que depende de fatores determinantes durante a construção correta e satisfatória da cultura empreendedora. Alguns fatores estão descritos na figura \ref{figura_2}.

\begin{figure}[H]
\centering
\caption{\textbf{Fatores que influenciam o aprendizado do comportamento empreendedor:}}
\includegraphics[scale=1]{Imagens/esquema_influencias_empreendedorismo.png}
\fonte{Adaptado de \cite{dornelas_como_2003}}
\label{figura_2}
\end{figure}


Segundo \citeonline{bacich_metodologias_2018} a aprendizagem mais profunda e efetiva, requer espaços de prática frequentes (aprender fazendo) e de ambientes ricos em oportunidades.  São importantes os estímulos interdisciplinares e multissensoriais, tais como o empreendedorismo acadêmico, utilizando para isto as metodologias ativas. Os mesmos autores definem as metodologias ativas como:

\begin{citacao}
Estratégias de ensino centradas na participação efetiva dos estudantes na construção do processo de aprendizagem, de forma flexível, interligada e híbrida. As metodologias ativas, num mundo conectado e digital, expressam-se por meio de modelos híbridos, com muitas possíveis combinações. A junção de metodologias ativas com modelos flexíveis e híbridos traz contribuições importantes para o desenho de soluções atuais para os aprendizes de hoje \cite{bacich_metodologias_2018}.
\end{citacao}

O ensino ativo do empreendedorismo acadêmico, apresenta-se também como uma potencial ferramenta para difusão e transferência de inovação e pesquisa realizadas por acadêmicos oriundos de laboratórios ou, departamentos onde a tecnologia se originou, \cite{guo_what_2019, abreu_nature_2013}, numa busca de oportunidades e iniciativas utilizando os meios existentes no ambiente acadêmico. Os conteúdos necessários ao efetivo ensino do empreendedorismo vão além da oferta de apenas uma disciplina, sendo preciso que a instituição de ensino, a partir de novas práticas pedagógicas, transforme-se em também em uma instituição empreendedora \cite{campelli_empreendedorismo_2011}, É necessário dar visibilidade a educação e promoção do comportamento empreendedor ao aluno com vistas a resolutividade de problemas \cite{degen_o_1989} e despertar da criatividade.

Diante da necessidade de solidificar o ensino empreendedor, que a \textit{Commission Enterprise and Industry Directorate-General} \cite{european_commission_best_2008} estruturou a educação empreendedora direcionada ao ensino superior em três objetivos, representados no modelo esquemático que pode ser visto na Figura \ref{figura_3}, em que se explica às três bases que estruturam os objetivos do ensino do Empreendedorismo no meio acadêmico. 

\begin{figure}[H]
\centering
\caption{\textbf{Pilares dos objetivos do ensino ao empreendedorismo.}}
\includegraphics[scale=0.8]{Imagens/objetivos_educacao_empreendedora.png}
\fonte{Adaptado de \cite{european_commission_best_2008}}
\label{figura_3}
\end{figure}

Como visto, ensino do empreendedorismo perpassa por diversas vertentes, porém, visando a associação de tais conteúdos aos técnicos científicos de forma interdisciplinar, é explorado neste trabalho a proposta de educação empreendedora que tem como base a Aprendizagem Baseada em Problemas (ABPR), a Aprendizagem Baseada em Projetos (ABP) \cite{bender_aprendizagem_2015} , que são os passos para elaboração dos conteúdos para o desenvolvimento de um empreendimento bem-sucedido segundo \cite{aulet_empreendedorismo_2019} no livro: Empreendedorismo Disciplinado. 
Segundo \cite{bender_aprendizagem_2015}a ABP tem como objetivo o desenvolvimento do autoconhecimento com ênfase na perseverança, na imaginação, na criatividade, na inovação, para resolubilidade de problemas reais. É muito importante o conteúdo que se aprende a fazer, mas, sobretudo, o que aprendido \cite{souza_disseminacao_2001}, de forma que a união de tais conhecimentos se some a um melhor desenvolvimento aos profissionais graduados que irão ao mercado de trabalho ou ao mundo dos negócios. 

O conhecimento científico promovido de forma interdisciplinar na graduação, além de repassar os conhecimentos técnicos, promove uma considerável contribuição para se desenvolver o raciocínio independente, criativo e inovador. Nesse sentido buscamos nesta pesquisa uma abordagem que possa explorar todos os conteúdos de uma metodologia disciplinada e estruturada.


\section{Propriedade Intelectual no meio rural}

A criatividade presente na mente humana gerou inúmeras, reformulações da natureza a partir de resoluções de problemas que surgem a partir da evolução das interação social, ela surge no meio de uma inquietação relacionada com algum problema existente no território em que a inovação está inserida. \cite{pacheco_dos_2018}. Desta forma, atualmente toda criação advinda do intelecto humano, tais como música, produto, processo, nova cultivar, desenhos, artigos científicos, trabalhos literários ou artísticos constituem um ativo, bem ou direito \cite{costa_interseccao_2011}. Tais ativos sendo eles registrados ou não, estão ligados diretamente ao seu autor/inventor sendo isto considerado direito de propriedade  intelectual. Este direito vai muito além da inovação em si \cite{wipo_tratado_1970}, mas a relação entre o seu autor/inventor e sua respectiva criação intelectual. 

Como forma de proteção aos possíveis direitos econômicos oriundos das produções intelectuais torna-se desejável que essas inovações e tecnologias sejam depositadas e patenteadas quando tais inovações satisfazem as requisições de registro industrial. Marcadamente quando pertencente aos campos de invenção, ou publicação de ideias e premissas e quando tais inovações pertencerem aos campos de registro de Direito Autoral \cite{wipo_b06_2019}. 

Aliadas  às  ações  para  o  desenvolvimento  de  inovações, o inventor ou autor deve ter certa atenção as Propriedades Intelectuais que tangem seu produto, ou processo. Sendo assim, é visível que há uma grande afinidade entre estes dois temas, principalmente os direitos que cabe ao meio industrial e do autor ligados a software. 

A crescente interação comercial,  financeira  e  tecnológica  entre  a  economia  e  seus agentes  exigem  padrões  modernos  de  proteção  para  a  PI,  uma  vez  que  os  direitos sobre os nomes empresariais, tecnologias, \textit{designs}, marcas, entre outros, representam valiosos ativos das empresas e profissionais autores/inventores \cite{sherwood_propriedade_1992}.  

Nesse contexto, o sistema de propriedade intelectual permite incentivar a geração de novas tecnologias, produtos e processos, tal como, promover a criação de empresas inovadoras em todas as áreas de informação, uma vez que permite ao autor/inventor explorar economicamente suas produções. Da mesma forma que os países com melhor proteção de Direito de Propriedade Intelectual (DPI), atraem significativamente mais atividades de fusões e aquisições transnacionais de alta tecnologia, particularmente nas economias em desenvolvimento, proporcionando um ganho maior de aporte estrangeiro para desenvolvimento tecnológico interno \cite{hasan_impacts_2017}, e registro de novas patentes. 

A patente e as publicações científicas, representam instrumentos jurídicos de proteção do invento resultante de um esforço de pesquisa dos institutos de pesquisa, e de pesquisadores no desenvolvimento de processos tecnológicos e novos produtos. Isto torna um investimento seguro, rentável e legítimo se prevenindo do comportamento desleal que podem surgir da concorrência, evitando a comercialização resultante de cópia não permitida, aspecto que desrespeita o esforço realizado pelos detentores originas e os gastos implicados no desenvolvimento \cite{marques_natureza_2017}.

O conceito de Propriedade Intelectual é amplo e difere de país a país, em suma pode ser entendido como, à área do Direito que, garante a inventores ou responsáveis por qualquer produção do intelecto, seja nos domínios industrias, científicos, literários ou artísticos, o direito de obter, por um determinado período de tempo, recompensa pela própria criação \cite{aspi_aspi_2019}.

O princípio de independência dos direitos inseridos no ordenamento jurídico internacional, possibilita que matéria protegida seja considerada de domínio privado podendo converter-se em um produto ou processo com direitos outorgados no país registrado, e pelo qual os demais interessados terão de pagar direitos de uso \cite{galvao_direitos_2002}. Se mostra importante a busca da compreensão atual das propriedades intelectuais e inovações tecnológicas à luz das áreas derivadas dos contextos dos recursos hídricos, a fim de que seja possível compreender o contexto e caminho do desenvolvimento tecnológico. 

O amparo legal e necessário para garantir que os investimentos em pesquisa e desenvolvimento retornem ao inventor, provocando um processo cíclico positivo, em que maiores investimentos em PD seriam promovidos diante da concessão do monopólio temporário de exploração do invento \cite{lima_sauglobal_2017}. No Brasil a PI é composta por três grandes áreas: Propriedade Industrial, Direito do Autor e Suis generis, demais ramificações podem ser vistas na Figura \ref{figura_4}.


\begin{figure}[H]
\centering
\caption{\textbf{Modalidades da propriedade intelectual no Brasil}}
\includegraphics[scale=0.4]{Imagens/propriedade_intelectual.png}
\fonte{Adaptado de \cite{inpi_manual_2017}}
\label{figura_4}
\end{figure}


Os documentos que compõem os pedidos de patente são constituídos por diferentes partes: relatório descritivo, reivindicações e resumo \cite{inpi_diretrizes_2011}. Por sua natureza redacional, deve descrever todo o objetivo, funcionalidades e reivindicações das proteções solicitadas, se mostrando um texto capaz de demonstrar quais os aspectos da invenção e quais os pontos inéditos que trazem esta novidade industrial, expressando desta forma possíveis inovações \cite{wipo_global_2018}.  


Diante de tais mudanças sobre o cenário do desenvolvimento tecnológico e das Propriedades Intelectuais (PI) inúmeras questões sobre o papel que os sistemas de registro e divulgação para PI desempenham no desenvolvimento e o incentivo à inovação \cite{segala_os_2016}. Os países que apresentam uma economia mais forte, dispõe de um sistema de proteção de propriedade mais robusto e confiável, concomitantemente, uma maior quantidade de registros e depósitos das mais variadas finalidades \cite{mueller_universidades_2014}.



