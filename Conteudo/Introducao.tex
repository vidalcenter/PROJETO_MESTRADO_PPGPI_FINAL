\chapter{INTRODUÇÃO}

O empreendedorismo  pode ser compreendido como a arte de fazer acontecer com criatividade e motivação. As vezes é visto como um vislumbre de esperança que pode trazer alguma contribuição para a melhoria da dinâmica e do desempenho econômico, especificamente na criação de emprego para profissionais recém formados, em geral, pela vivencia em treinamentos educacionais adicionais, que visam conduzir os novos profissionais ao alcance da visão estratégica de negócios, capacidade de liderança, e tornando estes, mais bem preparados para trabalhar com novas tecnologias, já que  perfil este requerido pelas grandes empresas e startups no agronegócio. No entanto, o empreendedorismo no setor agrícola é, em certas circunstâncias, visto como algo incompatível ou, pelo menos, difícil de implementar e ministrar.

Mais estudos científicos nos campos de educação direcionada ao desenvolvimento da intenção empreendedora  podem fornecer contribuições interessantes no caminho para destacar novas ideias dentro do setor agrícola. Nesse contexto, o  objetivo desta pesquisa , identificar de forma analítica a inovação sustentável e a eficácia da promoção empreendedora por meio de uma ação de educação com vistas aos negócios rurais, tendo como ferramenta promotora o Programa Empreenda Agro Sustentável, utilizando para este fim, Workshops encadeados e ferramentas didáticas que buscam capacita-los para a construção de propostas inovadoras viáveis.



\section{JUSTIFICATIVA}

Acredita-se que as oportunidades de trabalho para os novos dos profissionais atuantes na área das ciências agrárias vêm mudando, se transformando numa busca pela valorização das capacidades e competências ocupacionais. Busca-se cada vez mais a autoeficácia, a promoção de direitos de cidadania, associatividade política, responsabilidade social e ambiental, consideração, respeito às diversidades étnicas e culturais que devem ser valorizadas constantemente, e para tal, a academia tem um papel importante neste contexto, que é o de fomentar e oportunizar o surgimento destas competências. Dentro deste contexto a capacidade de manter-se fio a inovação, especialmente a disruptiva é fundamental ao progresso do crescimento e manutenção futura do profissional das ciências agrárias no novo mercado de serviços e produtos. 

De acordo com \citeonline{tarapanoff_monitoramento_2016}, existe um cenário favorável para os negócios rurais que buscam a sustentabilidade econômica e ambiental. Nesse sentido o Brasil deve buscar trilhar um caminho seguro em relação à sustentabilidade do agronegócio. Com efeito, em consonância com as melhores práticas da exploração ambiental e a produção agrícola, parte promovida por cumprimentos a regras ambientais a exemplo da Agenda 2030, que prevê a exploração consciente e sustentável dos recursos naturais e, tomando medidas urgentes sobre a mudanças climáticas individuais quanto institucionais, \cite{filho_documentos_2017}.

O desenvolvimento sustentável pode ser definido segundo \cite{lara_ideologia_2017} como um negócio socialmente responsável e ecologicamente correto, mas invariavelmente viável em termos financeiros. Concomitantemente existe uma lacuna na formação profissional durante o ensino superior no que se refere à adoção de uma cultura empreendedora, \cite{lima_ser_2015}. Não tem sido possibilitado aos acadêmicos, a oportunidade de gerar inovação tecnológica sustentável, inclusive para a aplicação prática dos conhecimentos adquiridos. Da mesma maneira, para que para uma ideia inovadora alcance o sucesso desejado é preciso muito mais que o conhecimento técnico.Deve ser disponibilizado aos futuros profissionais/empreendedores, treinamento na formulação das ideias em etapas direcionadas e adequadamente orientadas.

Para que um negócio sustentável possa ter sucesso é preciso mais do que uma ideia inovadora, deve ter o meio e os profissionais capacitados para tal. O Relatório da Fundação \textit{Wing Marion Kauffman} afirma que as Startups criam uma média de 3 milhões de novas vagas de empregos anualmente e que, estes tipos de empreendimentos serão responsáveis pela criação de 60\% dos empregos no  mundo  \cite{brasil_o_2017}. 
Atualmente estes tipos de  negócios contribuem para o crescimento de diversas regiões geográficas, já que não se expandem apenas em tamanho, mas também em novos locais, além de incentivar o emprego em suas indústrias relacionadas. Supletivamente, como muitas dessas microempresas são responsáveis por desenvolver novas tecnologias e processos, elas também geram aumento de absorção do capital humano mais capacitado para gerenciamento empresarial.

Diante deste cenário, para que o aprendizado dos profissionais seja mais efetivo, surgem diversas abordagens e metodologias a serem assimiladas. Nesse contexto, deve existir uma maior produção de estudos e conteúdos sobre o empreendedorismo e os modelos educacionais que melhor se apliquem ao aprendizado deste, como ressalta \citeonline{kuratako_entrepreneurship_2003}. É notória a urgência de se pesquisar o ensino em empreendedorismo de forma disciplinada no meio acadêmico. Por ser um tema de grande importância, a educação empreendedora promovida no seio do ensino superior pode ser o caminho para o surgimento de inovações sustentáveis e economicamente viáveis passives e escaláveis.

Dentro do contexto metodológico educacional, temos as metodologias ativas, que trazem a possibilidade de mudança da centralidade no docente (ensino) para o estudante (aprendizagem), ponto de vista preconizado por \citeonline{freire_pedagogia_1987} ao abordar educação como um processo que não é realizado por outrem, ou pelo próprio indivíduo, mas que acontece na interação entre pessoas através de sua vivência por palavras, ações e reflexões. 
Enquanto o método tradicional de ensino utiliza a transmissão de informações e concentra as atividades no docente, na metodologia ativa, os alunos ocupam a centralidade da educação e o conhecimento é construído de forma colaborativa. Sucintamente, as metodologias ativas propõem transformar o processo de ensinagem na busca pelo comportamento empreendedor, como uma forma de enfrentar o modelo tradicional praticado e aceito ao longo dos anos.
 
As práticas ativas estimulam o reconhecimento das dificuldades do mundo atual, tornando os alunos aptos a intervir na promoção das transformações necessárias a exemplo daquelas que se baseiam na reflexão e argumentação \cite{bezanilla_methodologies_2019}. Assim, o aluno torna-se protagonista da sua aprendizagem e autônomo no alcance dos seus objetivos incorporando seus valores e razões \cite{rubel_student_2016}. 

Existem vários recursos, métodos e técnicas para alcançar o satisfatório comportamento empreendedor, como: uso de tecnologias digitais e aplicativos \cite{pereira_use_2020}, ensino híbrido e suas estratégias como sala de aula em rotação por estações, Aprendizagem Baseada em Problemas (ABP) \cite{souza_aprendizagem_2015}, e uso de situações-problema e estudos de hipóteses problemas, sala de aula invertida \cite{junior_sala_2016,branco_sala_2016}, uso de mapas mentais \cite{junior_percepcao_2018}, sala de aula compartilhada \cite{strack_por_2009}, estratégias de Design Thinking \cite{andrews_circular_2015}, Gamificação \cite{ogawa_avaliacao_2016}, projetos de extensão \cite{garcia_contribuicao_2012}, dentre tantos outros recursos do método ativo, os quais podem vir facilitar o entendimento e a compressão dos acadêmicos das Ciências Agrárias no contexto de um mercado de trabalho que se apresenta com um perfil voltado ao empreendedor.

O empreendedorismo é a habilidade de reunir esforços para transformar em realidade uma oportunidade, objetivando a satisfação pessoal do empreendedor e o lucro. Tal conceito define o empreendedorismo como uma prática constante das atividades rotineiras dos educandos. Desde a capacidade de resolução de problemas quanto a idealização de propostas capazes de inovar. Dentro desta dicotomia entre empreendedorismo e educação surge a "Educação Empreendedora", que construída por práticas e dinâmicas idealizadas, buscando a melhoria na promoção do comportamento empreendedor \cite{martins_educacao_2016, morais_empreendedorismo_2018}, e resolução de problemas de forma sustentável e rápida.

O objetivo desse trabalho foi identificar de forma analítica a inovação sustentável e a eficácia da promoção empreendedora por meio de uma ação de educação com vistas aos negócios rurais, tendo como ferramenta promotora o Programa Empreenda Agro Sustentável. utilizando para este fim, Workshops encadeados e ferramentas didáticas que buscam capacita-los para a construção de propostas inovadoras viáveis.


\section{DELIMITAÇÕES DO ESTUDO}

Esta pesquisa está focada na dissonância entre a teoria e prática dos métodos educacionais e as grandes e continuas mudanças do mercado de trabalho no meio rural. Este Setor foi escolhido por estar contribuindo significativamente para a balança comercial do país, apresentando saldos positivos frequentes. Igualmente contribui, para a segurança alimentar do País e produção de produtos limpos e renováveis. O mercado emergente apresenta significativa contribuição para a empregabilidade da população no campo, invertendo cada vez mais o êxodo rural, porém, este mercado que absorve novos profissionais, exige que tais profissionais se mostrem a cada dia mais capacitados para lidar com o desenvolvimento tecnológico e a produção em larga escala, em que a busca pela valorização das capacidades e competências profissionais aumenta a cada dia. 

Em contraponto o empreendedorismo atualmente se confunde com a Meritocracia. Tantos a meritocracia quanto o empreendedorismo caminham juntos no cerne do movimento de individualização no mundo das ocupações laborais \cite{costa_novo_2019}. Os dois Projetam imagens individuais de labor e sucesso, em que a capacidade individual somada às oportunidades gera resultados positivos junto ao mercado de trabalho. Porém, o Empreendedorismo derivado da educação empreendedora, proposto neste projeto, surge atrelado a técnicas e métodos capazes de facilitar e validar as propostas empreendedoras. Assim, sendo um programa que vise o incentivo às práticas empreendedoras de forma sistemática e coerente distingue dos resultados negativos, aqueles que estão atrelados a meritocracia e suas vantagens, a frente dos pares distintos.

Visando compreender o comportamento empreendedor nos alunos dos cursos do Centro
Ciências Agrárias Aplicadas CCAA da Universidade Federal de Sergipe-UFS, foi definida a população para esta pesquisa de 1.453 discentes dos cursos de graduação nas áreas das ciências agrárias da Universidade Federal de Sergipe (UFS): Engenharia Agronômica, Engenharia Agrícola, Zootecnia, Engenharia Florestal, Medicina Veterinária e Engenharia de Pesca, que refletem os dados contidos no relatório estatístico de matrículas 2017 da instituição. A amostra de fato trabalhada compreenderá 120 discentes que participaram do Programa Empreenda Agro Sustentável.

As atividades foram desenvolvidas em quatro workshops, que trabalharam metodologias ativas, oficinas, palestras tendo em conta a promoção da aprendizagem significativa e colaborativa. Durante os módulos do projeto (workshops), os participantes testaram seus insights para que novas requisições fossem realizadas e/ou que erros nos planejamentos fossem encontrados e, consequentemente, debatidos e mitigados. Depois que todas as Sprints (atividades dos três workshops) foram finalizadas, ou seja, que todos os módulos foram abordados, foi iniciado um ciclo de Apresentações e desenvolvimento da habilidade de apresentação e demonstração dos produtos por apresentações sumárias (Pitchs). Em tempo, o programa foi desenvolvido em quatro encontros "workshops"(Figura 17), que abordaram temas pertinentes ao empreendedorismo e ao comportamento empreendedor, a saber:



\begin{itemize}

\item {1º Workshop INSIGHT E DESPERTAR: O que são startups, empreendedorismo, comportamento empreendedor e cultura empreendedora, problemas (segmentação do mercado), segundo os Objetivos do Desenvolvimento Sustentável (ODS), Modelagem do negócio e Criatividade;}
\item {2º Workshop IDEAÇÃO: A busca de oportunidades como característica empreendedora, construção do \textit{Lean Canvas}, mapa de empatia, validação da proposta de valor, economia colaborativa e \textit{coworking};}

\item {3º Workshop: Hackathon: Prototipagem para o MCVP, O que você pode ser feito por seu cliente e como o cliente adquire seu produto?;}
\item {4º Workshop Demoday: Destaca-se que o mecanismo de pesquisa que será utilizado neste experimento foi composto por Cinco blocos de questões de múltipla escolha baseadas principalmente em escalas de cinco ou sete possibilidades.}
\end{itemize}

O estudo foi caracterizado como uma pesquisa de levantamento ou Survey, que se destaca por compreender uma amostra expressiva em relação ao universo pesquisado \cite{freitas_o_2000}. Optou-se por adotar a abordagem quantitativa na mensuração dos resultados educacionais do Programa Empreenda Agro Sustentável. Após a aplicação dos instrumentos de análise, foi realizada a categorização dos dados para que fosse possível a classificação da pontuação adotada, segundo o estudo GUESS \cite{lima_educacao_2015} que utiliza testes de hipóteses sobre uma proporção. O projeto traz como benefícios: 

\begin{itemize}
\item{Promover o desenvolvimento pessoal, econômico, social no meio rural através da oportunidade de acesso a alternativas de produção de renda;}
\item{Criar a oportunidade de se trabalhar com o que realmente gosta e vencer os entraves do mercado econômico;}
\item{Dar autonomia e liberdade para conduzir o próprio talento, porém, orientado por metodologias específicas;}
\item{Transmitir valores e inspirar novos empreendedores no ambiente agrário;}
\item{Ensinar como lidar com os fracassos e frustrações, sabendo como os contornar;}
\item{Ensinar estratégias de organização de ideias ou carreiras buscando a receita positiva;}
\end{itemize}




\section{OBJETIVOS}

\subsection{OBJETIVO}

Identificar de forma analítica a inovação sustentável e a eficácia da promoção empreendedora por meio de uma ação de educação com vistas aos negócios rurais, tendo como ferramenta promotora o Programa Empreenda Agro Sustentável.



\subsection{OBJETIVOS ESPECÍFICOS}

\begin{itemize}
\item{Identificar os avanços na compreensão sobre empreendedorismo ao longo do andamento do Programa;}
\item {Avaliar o potencial empreendedor dos alunos do Centro de Ciências Agrárias Aplicadas participantes do Programa;}
\item {Fomentar por meio do projeto de extensão o comportamento empreendedor nos alunos do Centro de Ciências Agrárias Aplicadas;}
\item {Tipificar os tipos de PI que surgem com o incentivo ao empreendedorismo sustentável por meio da aplicação das metodologias trabalhadas no Programa.}

\end{itemize}


\section{PROBLEMA}

O comportamento empreendedor como indutor de inovação, pode ser estimulado mediante o uso de projetos de extensão universitária como o Programa Empreenda Agro Sustentável? 


\section{HIPÓTESE}

O Programa Empreenda Agro Sustentável ao despertar o comportamento empreendedor potencializa também a inovação entre os seus participantes.







%%%%%%%%%%%%%%%%%%%%%%%%%%%%%%%%%%%%%%%%%%%%%%%%%%%%%%%%%%%%%%%%%%%%%%%%%%%%%%%%%%%%%%%%%%%%%%%%%%%%%%%%%%%%%%%%%%%%%%%%%%%%%%%%%%%%%%%%%%%%%%%%%%%%%%
                                                                 %REFERENCIAL TEÓRICO%                                                                             
%%%%%%%%%%%%%%%%%%%%%%%%%%%%%%%%%%%%%%%%%%%%%%%%%%%%%%%%%%%%%%%%%%%%%%%%%%%%%%%%%%%%%%%%%%%%%%%%%%%%%%%%%%%%%%%%%%%%%%%%%%%%%%%%%%%%%%%%%%%%%%%%%%%%%%

