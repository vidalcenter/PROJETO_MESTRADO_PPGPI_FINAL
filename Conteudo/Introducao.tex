\chapter{INTRODUÇÃO}

As oportunidades de trabalho para os profissionais das ciências agrárias têm passado por transformações, numa busca pela valorização das capacidades e competências ocupacionais. Busca-se cada vez mais a autoeficácia, a responsabilidade social e ambiental, o respeito ás diversidades étnicas e culturais num movimento de valorização permanente. Nesse sentido a academia tem um papel importante, que é o de fomentar e oportunizar o surgimento destas competências. Dentro deste contexto a capacidade de estimular a inovação, especialmente a disruptiva associada ao empreendedorismo, é fundamental para o processo de crescimento e atuação futura do profissional das ciências agrárias no novo mercado de serviços e produtos. Em tempo, o ato de empreender pode ser compreendido como a arte de fazer acontecer com criatividade e motivação, ou seja, consiste no prazer de realizar com sinergismo e inovação qualquer projeto pessoal ou organizacional, em desafio permanente às oportunidades e riscos, e assumir um comportamento proativo diante de questões que precisam ser resolvidas.

O empreendedorismo quando relacionado a invetividade não consiste apenas na criação de empresas, ser empreendedor vai além de ter seu próprio negócio, implica em uma visão de mundo diferente, uma mudança de paradigma e de pensamento, podendo ser compreendido também como a arte de fazer acontecer com criatividade e motivação. As vezes este comportamento é visto como um vislumbre de esperança que pode trazer alguma contribuição para a melhoria da dinâmica e do desempenho econômico, especificamente na criação de emprego para profissionais recém-formados. 

O empreendedor é um constante inovador, sendo aquele que está sempre em busca de soluções, que consegue enxergar nas oportunidades, tendo iniciativa e sendo proativo e visionário. Os empreendedores, geralmente, buscam treinamentos educacionais adicionais que visam aprimorar a visão estratégica de negócios, capacidade de liderança, além de se prepararem para trabalhar com novas tecnologias, uma vez que este é o perfil requerido pelas grandes empresas do agro, empresas que dependem diretamente da inovação para se manterem competitivas. No entanto, o empreendedorismo no setor agrícola é, em certas circunstâncias, visto como algo incompatível ou, pelo menos, difícil de implementar e ministrar. Mostra-se necessário a promoção da cultura empreendedora nos futuros profissionais, para que tais profissionais possam contribuir para introduzir inovações na empresa na qual trabalha, ajudar a solucionar problemas da propriedade rural, entre outras possibilidades que se fazem presentes a partir do momento que um comportamento empreendedor e a cultura da inovação são desenvolvidos. Desta forma, o desenvolvimento de uma visão empreendedora como ferramenta a inovação é essencial para formação de profissionais que tenham iniciativa, visão estratégica e capacidade de liderança, perfil este requerido pelas grandes empresas e negócios autônomos. Além das características já citadas, é imprescindível para o profissional empreendedor e inovador desenvolver sua criatividade, pois assim, torna-se mais fácil inovar. 

A inovação, a propagação da inovação e o surgimento de novos empreendimentos em muitos países, são tidas como importantes sinais para o crescimento e recuperação de crises econômicas. A inovação é orientada de acordo com várias racionalidades, podendo ser observada por diversas óticas e utilizando diversos instrumentos para aprendizado. Com efeito, o ambiente acadêmico se apresenta como uma unidade básica para o desenvolvimento de novos processos inovadores onde tais conteúdos devem ser amplamente explorados. Para que haja inovação é necessário múltiplas habilidades, em pensar e agir diferente, encontrar soluções alternativas para os problemas e buscar ideias que tragam melhorias. 

Quando se fomenta a criatividade no ambiente educacional a conquista da autonomia é consequência, assim como também a adoção de uma postura empreendedora. Entretanto, na maioria das vezes, temos observado que essa importante característica não está sendo adequadamente desenvolvida no meio acadêmico. Ainda, as metodologias tecnicistas tradicionais de ensino exercem uma forte influência, as quais utilizam a transmissão de informações e concentram as atividades no docente, Tais posturas, não vem colaborando para o despertar da criatividade dos alunos, além de criar uma lacuna na formação profissional quanto ao desenvolvimento de competências essenciais. Porém este ambiente tem mudado, a partir do uso de metodologias ativas, as quais estimulam o reconhecimento dos problemas atuais, fortalecendo a criação de novos produtos, soluções e a dinamização de atividades diversas, se tornando uma oportunidade educacional de promoção do empreendedorismo, potencializando a universidade para a criação de mecanismos que fomentem a manifestação dessas competências.

As metodologias ativas procuram criar situações de aprendizagem que exercitem a cultura empreendedora, podem contribuir para o conhecimentos em ação, construir conhecimentos sobre os conteúdos envolvidos nas atividades que são propostas. Ademais, estas metodologias podem desenvolver estratégias cognitivas, capacidade crítica e reflexão sobre suas práticas, fornecem e recebem feedback, aprendem a interagir e exploram atitudes e valores pessoais e sociais. Atualmente são desenvolvidos diversos os recursos e ferramentas por meio dos quais as educação ativa pode se expressar: aulas expositivas, visitas e contato com o ambiente empresarial, desafios, estudos de situações problemas, construção da habilidade de aprender coletivamente. Este tipo de modelo pedagógico tem sido amplamente explorado pela educação em empreendedorismo. Este um contributo transversal de conteúdos usa da multidisciplinaridade em diferentes disciplinas e áreas não disciplinares consubstanciando em atividades ou projetos, desenvolvidos de forma participada por estudantes, profissionais e mentores colaborando para a transformação dos participantes, principalmente os estudantes. Ela de utiliza de diversas ferramentas das metodologias ativas para que, de maneira atrativa, promova o que de fato se busca nos futuros profissionais empreendedores. Nesta feita, estudos científicos no campo da educação direcionados ao desenvolvimento da educação em empreendedorismo podem fornecer contribuições interessantes no sentido de destacar novas ideias dentro do setor agrícola.

Em um mundo globalizado onde a tecnologia vem avançando e ganhando cada vez mais espaço, é essencial que o ensino não se restrinja às práticas tecnicistas tradicionais conteudistas, assim como, a comunicação não deve acontecer apenas unilateral, ela deve ter o aluno sendo o núcleo do aprendizado. Faz-se necessário o desenvolvimento no estudante de um senso crítico, indagador, e proativo, sendo a criatividade e demais competências essenciais ao crescimento acadêmico, pessoal e profissional, se tornando um ser autônomo. Nesse contexto, o objetivo desse trabalho foi identificar de forma analítica a inovação sustentável e a eficácia da promoção empreendedora por meio de uma ação de educação com vistas aos negócios rurais, tendo como ferramenta promotora o Programa Empreenda Agro Sustentável.



\section{JUSTIFICATIVA}

As oportunidades de trabalho para os novos profissionais atuantes na área das ciências agrárias vêm mudando, se transformando numa busca pela valorização das capacidades e competências ocupacionais. Busca-se cada vez mais a autoeficácia, a promoção de direitos de cidadania, associativismo político, responsabilidade social e ambiental, consideração, respeito às diversidades étnicas e culturais. Para tal, a academia tem um papel importante neste contexto, que é o de fomentar e oportunizar o surgimento destas competências. Dentro deste contexto, a capacidade de implementar inovação, especialmente a disruptiva é fundamental ao progresso do crescimento e manutenção da carreira do profissional das ciências agrárias no novo mercado de serviços e produtos. 


De acordo com \citeonline{tarapanoff_monitoramento_2016}, existe um cenário favorável para os negócios rurais que buscam a sustentabilidade econômica e ambiental. Nesse sentido, o Brasil deve buscar trilhar um caminho seguro em relação à sustentabilidade do agronegócio. Buscando, desta forma, consonância com as melhores práticas no uso dos recursos ambientais e a produção agrícola, cumprimento às regras ambientais, como por exemplo a Agenda 2030, que prevê o uso consciente e sustentável dos recursos naturais e, tomando medidas urgentes sobre as mudanças climáticas individuais quanto institucionais, \cite{filho_documentos_2017}. O desenvolvimento sustentável pode ser definido, segundo \cite{lara_ideologia_2017}, como um negócio socialmente responsável e ecologicamente correto, mas invariavelmente viável em termos financeiros.


Concomitantemente a esta realidade, existe uma lacuna na formação profissional durante o ensino superior dos estudantes de ciências agrárias no que se refere à adoção de uma cultura empreendedora, \cite{araujo_educacao_2019}. Não tem sido possibilitado aos acadêmicos a oportunidade de gerar inovação tecnológica sustentável, inclusive para a aplicação prática dos conhecimentos adquiridos. Da mesma maneira, para que uma ideia inovadora alcance o sucesso desejado é preciso muito mais que o conhecimento técnico. Deve-se ser disponibilizado aos futuros profissionais/empreendedores, treinamento na formulação das ideias em etapas direcionadas e adequadamente orientadas. Para que um negócio sustentável possa ter sucesso é preciso mais do que uma ideia inovadora, deve-se ter o meio e profissionais capacitados para tal. 


O Relatório da Fundação \textit{Wing Marion Kauffman} afirma que as Startups criam uma média de 3 milhões de novas vagas de empregos anualmente e que, estes empreendimentos serão responsáveis pela criação de 60\% das ocupações laborais no mundo \cite{brasil_neto_resistencia_2017}. Atualmente estas categorias de negócios contribuem para o crescimento de diversas regiões geográficas, já que não se expandem apenas em tamanho, mas também em novos locais, além de incentivar o emprego em suas indústrias relacionadas. Supletivamente, como muitas dessas microempresas são responsáveis por desenvolver novas tecnologias e processos, elas também geram aumento de absorção do capital humano mais capacitado para gerenciamento empresarial.


Diante deste cenário, para que o aprendizado dos profissionais seja mais efetivo, surgem diversas abordagens e metodologias a serem assimiladas. Nesse contexto, deve existir uma maior produção de estudos e conteúdos sobre o empreendedorismo e os modelos educacionais que melhor se apliquem ao aprendizado, como ressalta \citeonline{dionello_educacao_2020}. É notória a urgência de se pesquisar o ensino em empreendedorismo de forma disciplinada no meio acadêmico. Por ser um tema de grande importância, a educação em empreendedorismo promovida no seio do ensino superior pode ser o caminho para o surgimento de inovações sustentáveis e economicamente viáveis, passíveis e escaláveis.

No contexto metodológico educacional, temos as metodologias ativas, que trazem a possibilidade de mudança da centralidade no docente (ensino) para o estudante (aprendizagem). Os métodos compreendem a educação como um processo que não é realizado por outrem, ou pelo próprio indivíduo, mas que acontece na interação entre pessoas através de sua vivência por palavras, ações e reflexões \cite{paiva_metodologias_2016}. Enquanto o método tradicional de ensino utiliza a transmissão de informações e concentra as atividades no docente, na metodologia ativa, os alunos ocupam a centralidade da educação e o conhecimento é construído de forma colaborativa. Sucintamente, as metodologias ativas podem transformar o processo de ensino na busca pelo comportamento empreendedor, como uma forma de enfrentar o modelo tradicional praticado e aceito ao longo dos anos.
 
As práticas ativas estimulam o reconhecimento das dificuldades do mundo atual, tornando os alunos aptos a intervir na promoção das transformações necessárias, a exemplo daquelas que se baseiam na reflexão e argumentação \cite{bezanilla_methodologies_2019}. Assim, o aluno torna-se protagonista da sua aprendizagem e autônomo no alcance dos seus objetivos incorporando seus valores e razões \cite{rubel_student_2016}.

Existem vários recursos, métodos e técnicas para alcançar o satisfatório comportamento empreendedor, como: uso de tecnologias digitais e aplicativos \cite{pereira_use_2020}, ensino híbrido e suas estratégias como sala de aula em rotação por estações, Aprendizagem Baseada em Problemas (ABP) \cite{souza_aprendizagem_2015}, situações-problema e estudos de hipóteses problemas, sala de aula invertida \cite{junior_sala_2016,branco_sala_2016}, uso de mapas mentais \cite{junior_percepcao_2018}, sala de aula compartilhada \cite{strack_por_2009}, estratégias de Design Thinking \cite{andrews_circular_2015}, Gamificação \cite{ogawa_avaliacao_2016}, projetos de extensão \cite{santos_projeto_2019}. Dentre tantas outras ferramentas do método ativo que podem facilitar o entendimento e a compreensão dos acadêmicos das Ciências Agrárias no contexto de um mercado de trabalho que se apresenta com um perfil voltado ao empreendedorismo.

O empreendedorismo é a habilidade de reunir esforços para transformar em realidade uma oportunidade, objetivando a satisfação pessoal do empreendedor e o lucro. Tal conceito define o empreendedorismo como uma prática constante das atividades rotineiras dos educandos. Desde a capacidade de resolução de problemas quanto a idealização de propostas capazes de inovar. Dentro desta dicotomia entre empreendedorismo e educação surge a “Educação em Empreendedorismo”, que é construída por práticas e dinâmicas idealizadas, buscando a melhoria na promoção do comportamento empreendedor \cite{martins_educacao_2016, morais_empreendedorismo_2018}, e resolução de problemas de forma sustentável e rápida.

\section{DELIMITAÇÕES DO ESTUDO}

Esta pesquisa está focada na dissonância entre a teoria e prática dos métodos educacionais e as grandes e contínuas mudanças do mercado de trabalho no meio rural. Este setor foi escolhido por estar contribuindo significativamente para a balança comercial do país, apresentando saldos positivos frequentes. Igualmente contribui, para a segurança alimentar do País e produção de produtos limpos e renováveis. O mercado emergente apresenta significativa contribuição para a empregabilidade da população no campo, invertendo cada vez mais o êxodo rural, porém, este mercado que absorve novos profissionais, exige que tais profissionais sejam capazes de lidar com o desenvolvimento tecnológico e a produção em larga escala. 

Em contraponto, o empreendedorismo atualmente se confunde com a Meritocracia. Tanto a meritocracia quanto o empreendedorismo caminham juntos no cerne do movimento de individualização no mundo das ocupações laborais \cite{costa_novo_2019}. Os dois projetam imagens individuais de labor e sucesso, em que a capacidade individual somada às oportunidades gera resultados positivos junto ao mercado de trabalho. Porém, o Empreendedorismo derivado da educação em empreendedorismo, proposto neste projeto, surge atrelado às técnicas e aos métodos capazes de facilitar e validar as propostas empreendedoras,ou seja um programa que visa o incentivo às práticas empreendedoras de forma sistemática e coerente.

Visando compreender o comportamento empreendedor nos alunos dos cursos do Centro de Ciências Agrárias Aplicadas (CCAA) da Universidade Federal de Sergipe (UFS), foi definida a população para esta pesquisa de 1.453 discentes dos cursos do CCAA da UFS que refletem os dados contidos no relatório estatístico de matrículas 2017 da instituição, dos cursos: Engenharia Agronômica, Engenharia Agrícola, Zootecnia, Engenharia Florestal, Medicina Veterinária e Engenharia de Pesca. A amostra final compreendeu 118 discentes que participaram do Programa Empreenda Agro Sustentável.

As atividades foram desenvolvidas em quatro workshops, que trabalharam metodologias ativas, oficinas, palestras tendo em conta a promoção da aprendizagem significativa e colaborativa. Durante os módulos do projeto (workshops), os participantes testaram seus insights para que novas requisições fossem realizadas e/ou que erros nos planejamentos fossem encontrados e, consequentemente, debatidos e mitigados. Depois que todas as Sprints (atividades dos três workshops) foram finalizadas, ou seja, que todos os módulos foram abordados, foi iniciado um ciclo de apresentações e desenvolvimento da habilidade de apresentação e demonstração dos produtos por apresentações sumárias (Pitchs). 

O programa trouxe como principais benefícios: 

\begin{itemize}
\item{Promoção do desenvolvimento pessoal, econômico, social no meio rural através da oportunidade de acesso às alternativas de produção de renda;}
\item{Criação de oportunidade em trabalhar com o que realmente gosta e vencer os entraves do mercado econômico;}
\item{Autonomia e liberdade para conduzir o próprio talento, porém, orientado por metodologias específicas;}
\item{Valores e inspiração para os novos empreendedores no ambiente agrário;}
\item{Ensino de como lidar com os fracassos e frustrações, sabendo como os contornar;}
\item{Ensino de estratégias de organização de ideias ou carreiras buscando a receita positiva;}
\end{itemize}



Diversos experimentos sobre o empreendedorismo na América Latina têm utilizado \textit{Surveys} exploratórios realizados em residências ou com apelo direto aos donos de empresas para a coleta de dados, assim como em meios acadêmicos \cite{lima_ser_2015}. Desta forma, este estudo utiliza de um \textit{Survey} exploratório-descritivo para analisar o potencial do comportamento empreendedor e a competências Empreendedoras, dos acadêmicos dos cursos de graduação em Ciências Agrárias, inscritos no Programa Empreenda AGRO Sustentável. 

Foi utilizado como método de análise do fator indutor para melhoria o programa de extensão Empreenda Agro Sustentável o modelo Global \textit{Global University Entrepreneurial Spirit Students Survey} (GUESSS), conhecido nacionalmente por Estudo GUESSS. Esta ferramenta de ensaio acadêmico que busca caracterizar o espírito, as atividades e as intenções empreendedoras de estudantes universitários, de todos os níveis de aprendizagem e em todos os cursos universitários, bem como as condições de ensino e apoio a atividades empreendedoras.


A pesquisa GUESSS é realizada internacionalmente, e em 2018 alcançou mais de 208 000 estudantes de mais de 3000 universidades em 54 países inclusive no Brasil \cite{sieger_global_2018}. Seu principal objetivo é acompanhar indicadores perceptivos de variáveis de nível individual e contextual do ambiente universitário, relacionados ao empreendedorismo entre estudantes de nível superior.

As bases conceituais do Estudo GUESSS, se firma na teoria do comportamento planejado - TCP \cite{ajzen_perceived_2002}. Na concepção da TCP, a ação precede a intenção. \citeonline{lopes_jr_atitude_2005} afirma que a atitude aliada às normas subjetivas e a percepção de controle comportamental, irá formar o comportamento individual capaz de impulsionar o surgimento das atitudes empreendedoras.

Sua utilidade para se estudar a intenção empreendedora na esfera educacional, assim como a educação superior em empreendedorismo foi confirmada por muitas pesquisas anteriores a esta \cite{krueger_potencial_2018,gonzalez_predictors_2009,fayolle_effect_2006}. Ela viabiliza o estudo e a compreensão das diferentes atitudes que sustentam a intenção empreendedora, assim como o exame dos antecedentes que influenciam tais comportamentos \cite{lima_educacao_2014}. Desta forma, este estudo foi caracterizado como uma pesquisa de levantamento ou Survey, que se destaca por compreender uma amostra expressiva em relação ao universo pesquisado \cite{freitas_o_2000}. Optou-se por adotar a abordagem quantitativa na mensuração dos resultados educacionais do Programa Empreenda Agro Sustentável. Após a aplicação dos instrumentos de análise, foi realizada a categorização dos dados para que fosse possível a classificação da pontuação adotada, segundo o estudo GUESS \cite{meoli_how_2019} que utiliza testes de hipóteses sobre uma proporção populacional. 


\newpage

\section{OBJETIVOS}

\subsection{OBJETIVO GERAL}

Identificar de forma analítica a inovação sustentável e a eficácia da promoção empreendedora por meio de uma ação de educação com vistas aos negócios rurais, tendo como ferramenta promotora o Programa Empreenda Agro Sustentável.

\subsection{OBJETIVOS ESPECÍFICOS}

\begin{itemize}
\item{Identificar os avanços dos participantes na compreensão sobre empreendedorismo ao longo do andamento do Programa;}
\item {Avaliar o potencial empreendedor dos alunos do Centro de Ciências Agrárias Aplicadas participantes do Programa;}
\item {Fomentar por meio do projeto de extensão o comportamento empreendedor nos alunos do Centro de Ciências Agrárias Aplicadas;}
\item {Tipificar as categorias de Propriedade Intelectual que surgem com o incentivo ao empreendedorismo sustentável por meio da aplicação das metodologias trabalhadas no programa.}
\end{itemize}


\section{PROBLEMA}

O comportamento empreendedor como indutor de inovação, pode ser estimulado mediante o uso de projetos de extensão universitária como o Programa Empreenda Agro Sustentável? 


\section{HIPÓTESE}

O Programa Empreenda Agro Sustentável ao despertar o comportamento empreendedor potencializa também a inovação entre os seus participantes.



%%%%%%%%%%%%%%%%%%%%%%%%%%%%%%%%%%%%%%%%%%%%%%%%%%%%%%%%%%%%%%%%%%%%%%%%%%%%%%%%%%%%%%%%%%%%%%%%%%%%%%%%%%%%%%%%%%%%%%%%%%%%%%%%%%%%%%%%%%%%%%%%%%%%%%
                                                                 %REFERENCIAL TEÓRICO%                                                                             
%%%%%%%%%%%%%%%%%%%%%%%%%%%%%%%%%%%%%%%%%%%%%%%%%%%%%%%%%%%%%%%%%%%%%%%%%%%%%%%%%%%%%%%%%%%%%%%%%%%%%%%%%%%%%%%%%%%%%%%%%%%%%%%%%%%%%%%%%%%%%%%%%%%%%%