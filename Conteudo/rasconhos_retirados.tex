Definir o desenvolvimento rural sustentável requer um esforço observacional e prático, pois, este ambiente vem sofrendo profundas transformações em suas demandas e necessidades. O desenvolvimento que antes se apresentava majoritariamente como produção de subsistência, hoje dá lugar a um complexo sistema agroindustrial \cite{bastos_determinantes_2018} e social. É importante neste sentindo compreender que definir o desenvolvimento rural com apenas um conceito seria uma proposição simplista do contexto de crescimento no meio rural. Partindo da definição de consequência de ações governamentais definidas por \citeonline{navarro_desenvolvimento_2001} como “atitudes práticas”, este autor descreve que:

\begin{citacao}
[...] Desenvolvimento rural, portanto, pode ser analisado a “posteriori”, neste caso se referindo às análises sobre programas já realizados pelo Estado (em seus diferentes níveis) visando a alterar facetas do mundo rural a partir de objetivos previamente definidos. Porem pode se referir também à elaboração de uma ação prática.
\end{citacao}

O desenvolvimento rural também pode ser compreendido por um conceito mais regional definido como: “Desenvolvimento Local”. Tal expressão é recente e deriva de iniciativas de mobilização organização social no sentido de promover uma maior representação dos diferentes atores sociais no processo de desenvolvimento. O Estado assume papel de agente facilitador desse processo de descentralização das políticas públicas para ser democrático, buscar transparência de suas instituições, o equilíbrio das forças exercidas pelas diferentes correntes de interesse e o compromisso com a qualidade de vida na população afetada \cite{castro_agricultura_2017}.
Tal conceito demonstra o espaço rural como um local ideal para a promoção de políticas de inovação e a construção de padrões inovadores na relação entre populações e instâncias públicas, numa tentativa de rompimento com a dominação, que parte de baixo para cima. Neste contexto, surge as Organizações Não Governamentais (ONGs) que buscam garantir a participação da população local, e fazer valer tais mudanças atuando normalmente em ambientes geograficamente mais restritos (região rural, povoados ou municípios), \cite{assis_agricultura_2005, teixeira_o_2016}.