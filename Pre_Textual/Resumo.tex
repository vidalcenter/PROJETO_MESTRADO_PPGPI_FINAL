% resumo em português
\setlength{\absparsep}{18pt} % ajusta o espaçamento dos parágrafos do resumo
\begin{resumo}

Ser empreendedor não significa apenas ter seu próprio negócio, o empreendedorismo está ligado às ações e capacidades individuais para criar, modular e sustentar ideias inovadoras globalmente. Assim, as ideias normalmente estão ligadas à inovação, algo que transcende o comum. Algumas ferramentas tais como workshops, jornadas, cursos, palestras podem ser usadas para difusão do comportamento empreendedor tanto em órgãos privados como em instituições de ensino, de modo a incutir novas ações empreendedoras sustentáveis, no sentido comercial, nas áreas industriais e agrícolas no meio acadêmico. Faz-se necessário, então, a construção de um cenário favorável ao desenvolvimento da cultura empreendedora no ambiente acadêmico, principalmente nas ciências agrárias. Neste contexto, muitas nações estão buscando promover atividades empreendedoras, utilizando, para isto, programas educacionais promovidos nos meios acadêmicos alcançando um importante público jovem. Neste sentido, este estudo tem como objetivo identificar, de forma analítica, a inovação sustentável e a eficácia do incentivo ao empreendedorismo por meio de uma ação educacional com vistas aos negócios rurais, tendo como ferramenta promotora o Programa Empreenda AGRO Sustentável. O Programa visa buscar o engajamento dos alunos para o desenvolvimento de novos negócios direcionados ao setor agrícola, podendo este experimento também direcionar futuras pesquisas científicas focadas em produtos capazes de satisfazer às novas demandas, de forma eficiente e sustentável. A metodologia de natureza quantitativa exploratória e de estudo de caso foi utilizada para avaliar quantitativamente o Programa Empreenda AGRO Sustentável com coleta de dados por meio da técnica “survey” tendo como base o estudo GUESS. A ferramenta de promoção prática do comportamento empreendedor “Programa Empreenda AGRO Sustentável" foi conduzida buscando a disseminação do empreendedorismo junto aos acadêmicos do Centro de Ciências Agrárias Aplicadas da Universidade Federal de Sergipe (UFS). Quatro Workshops foram planejados e executados de maneira sistemática e interativa, em que foram aplicadas metodologias e ferramentas para modelagem de negócios como Design Thinking com o intuito de despertar ideias de startups por meio de oficinas interdisciplinares fomentando, desta forma, a construção de Mínimos Produtos Comercialmente Viáveis (MCVP). Por meio do Programa Empreenda Agro Sustentável e a partir da aplicação das metodologias ativas, 15 equipes discutiram e amadureceram ideias em modelos de startups, resultando em modelos de negócios escaláveis e negociáveis direcionados ao meio rural. Como resultado, a pesquisa constatou que os alunos participantes apresentaram forte influência positiva na intenção empreendedora e na autoeficácia para o empreendedorismo. Além disso, a pesquisa demonstrou que os alunos envolvidos no decorrer do programa evoluíram positivamente as dimensões empreendedoras estudadas, influenciando assim o desenvolvimento de novos negócios planejados na etapa de pré-aceleração, e encontraram maior segurança para o próximo passo que é o de aceleração de seus planos de negócios, conquistando autonomia, buscando novas oportunidades como profissionais proativos no mercado de trabalho.

\textbf{Palavras-chave}: Empreendedorismo; Competências empreendedoras; Negócios sustentáveis; Propriedade Intelectual; Projeto de Extensão.
\end{resumo}