% resumo em português
\setlength{\absparsep}{18pt} % ajusta o espaçamento dos parágrafos do resumo
\begin{resumo}

 %Melhorar esse resumo para projetos!!!

Ser empreendedor não significa apenas ter seu próprio negócio, o empreendedorismo está ligado às ações e capacidades individuais, ou seja, possuir características empreendedoras é ser capaz de criar, modular e sustentar ideias inovadoras globalmente, assim as ideias normalmente estão ligadas à inovação algo que transcende o comum. Algumas ferramentas tais como Workshops, jornadas, cursos, palestras podem ser usadas para difusão do comportamento empreendedor tanto em órgãos privados como em instituições de ensino. Faz-se necessário, a construção de um cenário favorável ao desenvolvimento da cultura empreendedora no ambiente acadêmico, principalmente nos centros das ciências agrárias, buscando o engajamento dos alunos perante o desenvolvimento de negócios direcionados ao setor agrícola, bem como direcionamento de futuras pesquisas científicas focadas em produtos capazes de satisfazer as novas demandas do meio agrário, deforma, eficiente e sustentável. Em razão dessa recorrente demanda, este estudo apresenta como questão principal de pesquisa: O comportamento empreendedor como indutor de inovação, pode ser estimulado mediante o uso de projetos de extensão universitária como o Programa Empreenda Agro Sustentável? Adicionalmente outras questões de pesquisa também se apresentam importantes: Qual o nível de conhecimento acerca do empreendedorismo pelos estudantes dos participantes do Programa? Quais as perspectivas de condução de ações empreendedoras a partir do aprendizado por meio do programa? Que mudanças em nível conceitual sobre empreendedorismo e comportamento empreendedor os alunos adquiriram? A metodologia de natureza qualitativa é utilizada para avaliar o Programa Empreenda AGRO Sustentável, com análise dos dados, que são coletados por meio da técnica quantitativa, um “survey” tendo como base o estudo GUESS. A ferramenta de promoção prática do exercício do comportamento empreendedor  “Programa Empreenda AGRO Sustentável", vem sendo conduzido buscando a disseminação do empreendedorismo junto aos acadêmicos do Centro de Ciências Agrárias Aplicadas da Universidade Federal de Sergipe - UFS, por meio de quatro Workshops, planejados de forma sistemática e interativa, sendo aplicadas desde ferramentas para modelagem de negócios como Design Thinking visando despertar ideias de startups em oficinas interdisciplinares fomentando desta forma a construção de Mínimos Produtos Comercialmente Viáveis (MCVP). É diante desta problemática que este projeto de pesquisa busca avaliar a inovação sustentável e a eficácia da promoção empreendedora por meio de uma ação de educação com vistas aos negócios rurais, tendo como ferramenta promotora o Programa Empreenda Agro Sustentável promovido para os acadêmicos das ciências agrárias, utilizando para este fim, Workshops encadeados e ferramentas didáticas que buscam capacita-los para a construção de propostas inovadoras viáveis.


 \textbf{Palavras-chave}: Negócios sustentáveis, Propriedade Intelectual, Projeto de Extensão.
\end{resumo}