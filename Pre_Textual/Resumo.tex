% resumo em português
\setlength{\absparsep}{18pt} % ajusta o espaçamento dos parágrafos do resumo
\begin{resumo}

Ser empreendedor não significa apenas ter seu próprio negócio, o empreendedorismo está ligado às ações e capacidades individuais, ou seja, possuir características empreendedoras é ser capaz de criar, modular e sustentar ideias inovadoras globalmente, assim as ideias normalmente estão ligadas à inovação algo que transcende o comum. Algumas ferramentas tais como workshops, jornadas, cursos, palestras podem ser usadas para difusão do comportamento empreendedor tanto em órgãos privados como em instituições de ensino, ou seja, incutir novas ações empreendedoras no sentido comercial, nas áreas industriais e agrícolas que sejam sustentáveis. Faz-se necessário, então, a construção de um cenário favorável ao desenvolvimento da cultura empreendedora no ambiente acadêmico, principalmente nas ciências agrárias. Para construção deste senário, muitas nações estão buscando promover a atividade empreendedora, utilizando para isto programa educacionais promovidos nos meios acadêmicos alcançando um importante público os jovens. Assim, torna-se evidente que o sistema educacional que promova o desenvolvimento das competências empreendedoras tem um importante papel a cumprir para com a causa empreendedora. Neste sentido, este estudo estudo tem como objetivo, identificar de forma analítica a inovação sustentável e a eficácia da promoção empreendedora por meio de uma ação de educação com vistas aos negócios rurais, tendo como ferramenta promotora o Programa Empreenda Agro Sustentável. O Programa visa buscar o engajamento dos alunos para o desenvolvimento de novos negócios direcionados ao setor agrícola, podendo este experimento também direcionar futuras pesquisas científicas focadas em produtos capazes de satisfazer às novas demandas, de forma eficiente e sustentável. A metodologia de natureza quantitativa exploratória e de estudo de caso, foi utilizada para avaliar quantitativamente o Programa Empreenda AGRO Sustentável, com análise dos dados, que foram coletados por meio da técnica quantitativa “survey” tendo como base o estudo GUESS. A ferramenta de promoção prática do exercício do comportamento empreendedor “Programa Empreenda AGRO Sustentável", foi conduzida buscando a disseminação do empreendedorismo junto aos acadêmicos do Centro de Ciências Agrárias Aplicadas da Universidade Federal de Sergipe – UFS. Foram conduzidos quatro Workshops, planejados de forma sistemática e interativa, sendo aplicadas ferramentas para modelagem de negócios como Design Thinking visando despertar ideias de startups em oficinas interdisciplinares fomentando desta forma a construção de Mínimos Produtos Comercialmente Viáveis (MCVP). Por meio do Programa Empreenda Agro Sustentável e a partir da aplicação das metodologias ativas, 15 equipes discutiram e amadureceram ideias em modelos de startups, resultando em modelos de negócios escaláveis e negociáveis direcionados ao meio rural. Pode-se perceber também que os alunos participantes apresentam forte influência positiva na intenção empreendedora e na autoeficácia para o empreendedorismo. Por meio dos resultados obtidos nesta pesquisa ficou evidente que, os alunos envolvidos no decorrer do programa evoluíram positivamente as dimensões empreendedoras estudadas, influenciando assim o desenvolvimento de novos negócios planejados na etapa de pré-aceleração, e encontram maior segurança para o próximo passo que é o de aceleração de seus planos de negócios, conquistando autonomia, buscando novas oportunidades, como profissionais proativos com destaque no mercado de trabalho.

\textbf{Palavras-chave}: Empreendedorismo; Competências empreendedoras; Negócios sustentáveis; Propriedade Intelectual; Projeto de Extensão.
\end{resumo}