% resumo em inglês
\setlength{\absparsep}{18pt} % ajusta o espaçamento dos parágrafos do resumo
\begin{resumo}[Abstract]
 \begin{otherlanguage*}{english}
   
Being an entrepreneur does not just mean having your own business, entrepreneurship is linked to individual actions and capabilities, that is, having entrepreneurial characteristics is being able to create, modulate and sustain innovative ideas globally, so ideas are usually linked to innovation, something that transcends the common. Some tools such as Workshops, seminars, courses, lectures can be used to disseminate entrepreneurial behavior both in private bodies and in educational institutions. It is necessary to build a favorable scenario for the development of entrepreneurial culture in the academic environment, mainly in the agricultural sciences. The idea is to seek student engagement in the development of business aimed at the agricultural sector, as well as directing future scientific research focused on products capable of satisfying new demands, in an efficient and sustainable manner. Due to this recurring demand, this study presents the main research question: Can entrepreneurial behavior as an innovation driver be stimulated through the use of university extension projects such as the Empreenda Agro Sustentável Program? In addition, other research questions are also important: What is the level of knowledge about entrepreneurship of students participating in the Program? What are the prospects for conducting entrepreneurial actions based on learning through the program? The objective of this work was to analytically identify sustainable innovation and the effectiveness of entrepreneurial promotion through an educational action aimed at rural businesses, using the Empreenda Agro Sustentável Program as a promoting tool. The qualitative methodology was used to evaluate the Empreenda AGRO Program
Sustainable, with data analysis, which were collected using the quantitative technique
Survey based on the GUESS study. The tool for the practical promotion of the exercise of entrepreneurial behavior "Programa Empreenda AGRO Sustentável", was carried out seeking the dissemination of entrepreneurship among academics at the Center for Applied Agricultural Sciences of the Federal University of Sergipe - UFS. Four Workshops were conducted, systematically planned and interactive, with business modeling tools such as Design Thinking being applied to awaken ideas from startups in interdisciplinary workshops, thus fostering the construction of Commercially Viable Minimum Products (MCVP).

 \textbf{Keywords}:Sustainable business, Intellectual Property, Extension Project.
 \end{otherlanguage*}
\end{resumo}