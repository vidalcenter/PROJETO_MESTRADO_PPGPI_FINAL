% resumo em inglês
\setlength{\absparsep}{18pt} % ajusta o espaçamento dos parágrafos do resumo
\begin{resumo}[Abstract]
 \begin{otherlanguage*}{english}
   
Being an entrepreneur doesn't just mean having your own business. Entrepreneurship is linked to the individual actions and capabilities that forge it. Possessing entrepreneurial characteristics and being able to create modular and sustain innovative ideas in their environment and in the world, such ideas are usually linked to innovation something that transcends the ordinary. Workshops, seminars, courses, lectures, everything and are being used as tools to disseminate entrepreneurial behavior, from private bodies to educational institutions.  It is necessary to build a favorable scenario for the development of entrepreneurial culture in the academic environment, especially the centers of agricultural sciences, seeking the engagement of students in the development of business directed to the agricultural sector, as well as the direction of future scientific research focused on products capable of satisfying the new demands of the agricultural environment in an efficient and sustainable way. Considering this discussion, this study will have as a research question: \textbf{Assess the entrepreneurial initiation aimed at innovation in rural areas through the Agro-Sustainable Enterprise program}, based on the following research questions: What is the level of knowledge about entrepreneurship at the center? What are the perspectives from the learning through the program? What changes at the conceptual level about entrepreneurship and entrepreneurial behavior have students acquired? The methodology of qualitative nature makes use of the case of the AGRO Sustainable Entrepreneurship Program from the Federal University of Sergipe-UFS. For the analysis of the data used the quantitative technique, a \textit{survey}, The surveys are surveys in which the obtaining of data can be done through sampling or groups with judgment criteria \cite{visser_survey_2000}, The questionnaire used is an adapted version of the data collection instrument of the international GUESSS study, translated and theoretically validated into Portuguese by experts in entrepreneurship and EE in Brazil. The program "Empreenda AGRO Sustentável", to instigate entrepreneurship to the academics of the Agricultural Sciences Center of the Federal University of Sergipe - UFS, through four workshops, planned in a systematic and interactive way, being applied from tools for business modeling to the construction of the Minimum Viable Product (MVP).

 \textbf{Keywords}: .
 \end{otherlanguage*}
\end{resumo}