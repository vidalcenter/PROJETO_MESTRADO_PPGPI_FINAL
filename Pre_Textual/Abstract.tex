% resumo em inglês
\setlength{\absparsep}{18pt} % ajusta o espaçamento dos parágrafos do resumo
\begin{resumo}[Abstract]
 \begin{otherlanguage*}{english}
   
Being an entrepreneur does not just mean having your own business, entrepreneurship is linked to individual actions and capabilities to create, modulate and sustain innovative ideas globally. Thus, ideas are usually linked to innovation, something that transcends the ordinary. Some tools such as workshops, seminars, courses, lectures can be used to disseminate entrepreneurial behavior both in private bodies and in educational institutions, in order to instill new sustainable entrepreneurial actions, in the commercial sense, in the industrial and agricultural areas in the academic environment. . It is necessary, then, to build a favorable scenario for the development of entrepreneurial culture in the academic environment, mainly in the agricultural sciences. In this context, many nations are seeking to promote entrepreneurial activities, using educational programs promoted in academic circles to reach an important young audience. In this sense, this study aims to identify, in an analytical way, sustainable innovation and the effectiveness of encouraging entrepreneurship through an educational action aimed at rural businesses, using the Empreenda AGRO Sustentável Program as a promoting tool. The Program aims to seek student engagement for the development of new businesses aimed at the agricultural sector, and this experiment can also direct future scientific research focused on products capable of satisfying new demands, in an efficient and sustainable manner. The methodology of quantitative exploratory nature and case study was used to quantitatively evaluate the Empreenda AGRO Sustentável Program with data collection using the “survey” technique based on the GUESS study. The tool for the practical promotion of entrepreneurial behavior “Programa Empreenda AGRO Sustentável” was conducted seeking the dissemination of entrepreneurship among academics at the Center for Applied Agricultural Sciences at the Federal University of Sergipe (UFS). Four workshops were planned and executed in a systematic and interactive way , in which methodologies and tools for business modeling such as Design Thinking were applied in order to awaken ideas from startups through interdisciplinary workshops, thus promoting the construction of Minimum Commercially Viable Products (MCVP). Sustainable and based on the application of active methodologies, 15 teams discussed and matured ideas in models of startups, resulting in scalable and negotiable business models aimed at the rural environment. As a result, the research found that the participating students had a strong positive influence on the intention entrepreneur a and self-efficacy for entrepreneurship. In addition, the research demonstrated that the students involved in the course of the program positively evolved the entrepreneurial dimensions studied, thus influencing the development of new businesses planned in the pre-acceleration stage, and found greater security for the next step, which is the acceleration of their business plans, gaining autonomy, seeking new opportunities as proactive professionals in the job market.

 \textbf{Keywords}:Sustainable business, Intellectual Property, Extension Project.
 \end{otherlanguage*}
\end{resumo}